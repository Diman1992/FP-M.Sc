% ========================================
%	Header einbinden
% ========================================

\input{longheader.tex}

% ========================================
%	Angaben für das Titelblatt
% ========================================

\title{Versuch V57 - Rauschen\\				% Titel des Versuchs 
\large TU Dortmund, Fakultät Physik\\ 
\normalsize Fortgeschrittenen-Praktikum}

\author{Jan Adam\\			% Name Praktikumspartner A
{\small \href{jan.adam@tu-dortmund.de}{jan.adam@tu-dortmund.de}}	% Erzeugt interaktiven einen Link
\and						% um einen weiteren Author hinzuzfügen
Felix Wieland\\					% Name Praktikumspartner B
{\small \href{felix.wieland@tu-dortmund.de}{felix.wieland@tu-dortmund.de}}		% Erzeugt interaktiven einen Link
}
\date{09.06.2015}				% Das Datum der Versuchsdurchführung

% ========================================
%	Das Dokument beginnt
% ========================================

\begin{document}

% ========================================
%	Titelblatt erzeugen
% ========================================

\maketitle					% Jetzt wird die Titelseite erzeugt
\thispagestyle{empty} 				% Weder Kopfzeile noch Fußzeile

% ========================================
%	Der Vorspann
% ========================================

%\newpage					% Wenn Verzeichnisse auf einer neuen Seite beginnen sollen
%\pagestyle{empty}				% Weder Kopf- noch Fußzeile für Verzeichnisse

\tableofcontents

%\newpage					% eine neue Seite
%\thispagestyle{empty}				% Weder Kopf- noch Fußzeile für Verzeichnisse
%\listoffigures					% Abbildungsverzeichnis

%\newpage					% eine neue Seite
%\thispagestyle{empty}				% Weder Kopf- noch Fußzeile für Verzeichnisse
%\listoftables					% Tabellenverzeichnis
\newpage					% eine neue Seite


% ========================================
%	Kapitel
% ========================================

%\section{Einleitung}				% Bei Bedarf

\section{Auswertung}
\subsection{Eigenrauschen}
Zunächst muss das Eigenrauschen der Apparatur bei verschiedenen Gain Einstellungen bestimmt werden, da alle folgenden Messungen um diesen Wert korrigiert werden müssen.

Dazu wurde ein \SI{0}{\ohm} Widerstand an den Eingang angeschlossen und anschließend das Rauschspannungsquadrat $U_0^2$ bei allen Nachverstärkungsstufen $(V_N)$ gemessen. Die Ergebnisse der Messungen stehen in den Tabellen \ref{tab:eigenRauschen_norm} und \ref{tab:eigenRauschen_korr}.

\begin{table}[h]
	\begin{subtable}[c]{0.3\textwidth}
		\begin{tabular}{lS}
			{$V_N$} & {$U_0^2$ [\si{\milli\volt}]}\\
			\toprule
			1	& 4.0\\
			2	& 4.0\\
			5	& 4.0\\
			10	& 4.0\\
			20	& 4.0\\
			50	& 4.0\\
			100	& 3.6\\
			200	& 1.0\\
			500	& 12.0\\
			1000& 55.6
		\end{tabular}
		\caption{Normale Schaltung}
		\label{tab:eigenRauschen_norm}
	\end{subtable}%
	\hspace{1cm}%
	\begin{subtable}[c]{0.3\textwidth}
		\begin{tabular}{lS}
			{$V_N$} & {$U_0^2$ [\si{\milli\volt}]}\\
			\toprule
			1	&-11.1 \\
			2	&-10.9 \\
			5	&-11.0 \\
			10	&-11.1 \\
			20	&-11.0 \\
			50	&-10.8 \\
			100	&-10.2 \\
			200	&-9.8 \\
			500	&0.5 \\
			1000	&53.0
		\end{tabular}
		\caption{Korrelationsschaltung}
		\label{tab:eigenRauschen_korr}
	\end{subtable}
	\caption{Eigenrauschen der Apparatur bei den beiden verwendeten Aufbauten.}
	\label{tab:eigenRauschen}
\end{table}

\subsection{Eichung}
Bevor die eigentlichen Messungen durchgeführt werden können, muss die Apparatur zudem geeicht werden.

Ein Sinuswellengenerator wird mit einer Amplitude von \SI{175}{\milli\volt} an den Eingang angeschlossen und durch einen Abschwächer um den Faktor 1000 reduziert. Der Bandpass wird so eingestellt, dass er nur Frequenzen im Bereich von \SIrange{10}{20}{\kilo\hertz} passieren lässt und die Nachverstärkung wird so eingestellt, dass die Spannungen mit genug Nachkommastellen im Millivoltbereich abgelesen werden kann. Die Messwerte stehen in Tabelle \ref{} und eine graphische Darstellung der mittels Gleichung \ref normierten Spannung $\eta(\nu)$ ist in Abbildung \ref zusehen.

\begin{figure}[h]
	\includegraphics[width=\textwidth]{../auswertung/results/sin.pdf}
	\caption{Messwerte aus der Eichmessung. Der Bandpass wurde auf das Intervall \SI{10}{\kilo\hertz} - \SI{20}{\kilo\hertz} eingestellt.}
\end{figure}

Um die Apparaturkonstante
\begin{align}
	\Delta\nu = \int_0^\infty \eta(\nu)\ d\nu
\end{align}
Zu bestimmen, wird die Integralfläche mittels Trapezregel bestimmt.
Damit ergibt sich für A:
\begin{align}
	\Delta\nu = 3870.61 \pm 
\end{align}
Der Fehler wurde mittels des Restgliedes der Trapezregel abgeschätzt:
\begin{align}
	\textbar R(f)\textbar = \frac{b-a}{12}h^2\text{max}\left(f''(x)\right)
\end{align}

Um die Boltzmann-Konstante $k_B$ zu berechnen, wird der Sinusgenerator durch einen Variablen Widerstand ausgetauscht und die Rauschspannung gegen den Widerstand aufgetragen. Mittels linearer Regression, kann dann $k_B$ bestimmt werden.

\begin{figure}
	\includegraphics[width=\textwidth]{../auswertung/results/wid1.pdf}
	\caption{Thermisches Rauschen Widerstand 1 (\SI{0}{\ohm} - \SI{100}{\ohm})}
	\label{fig:thermRauschen1}
\end{figure}

\begin{figure}
	\includegraphics[width=\textwidth]{../auswertung/results/wid2.pdf}
	\caption{Thermisches Rauschen Widerstand 2 (\SI{0}{\ohm} - \SI{1000}{\ohm})}
	\label{fig:thermRauschen2}
\end{figure}

Mittels der Niquist-Beziehung (Gleichung \ref) lässt sich aus den Steigungen der Ausgleichsgeraden die Boltzmannkonstante berechnen.

\begin{align}
	k_B = \frac{m}{4T\Delta\nu}
\end{align}

\vspace{2cm}
\textbf{Literatur}

\vspace{0.3cm}
[1] Anleitung des Lehrstuhlversuchs

% ========================================
%	Literaturverzeichnis
% ========================================

%\bibliographystyle{plainnat}			% Bibliographie-Style auswählen
%\bibliography{BIBDATEI}			% Literaturverzeichnis

% ========================================
%	Das Dokument endent
% ========================================

\end{document}
