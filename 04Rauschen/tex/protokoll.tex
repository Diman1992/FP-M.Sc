% ========================================
%	Header einbinden
% ========================================

\documentclass[bibtotoc,titlepage]{scrartcl}

% Deutsche Spracheinstellungen
\usepackage[ngerman,german]{babel, varioref}
\usepackage[T1]{fontenc}
\usepackage[utf8]{inputenc}

%\usepackage{marvosym}

\usepackage{amsfonts}
\usepackage{amssymb}
\usepackage{amsmath}
\usepackage{amscd}
\usepackage{amstext}
\usepackage{float}
\usepackage{caption}
\usepackage{wrapfig}
\usepackage{setspace}
\usepackage{threeparttable}
\usepackage{footnote}

\usepackage{caption}
\usepackage{subcaption}

\newfloat{formel}{htbp}{for}
\floatname{formel}{Formel}


\usepackage{longtable}

%\usepackage{bibgerm}

\usepackage{footnpag}

\usepackage{ifthen}                 %%% package for conditionals in TeX
\usepackage{siunitx}
%Fr textumflossene Bilder und Tablellen
%\usepackage{floatflt} - veraltet

%Fr Testzwecke aktivieren, zeigt labels und refs im Text an.
%\usepackage{showkeys}

% Abstand zwischen zwei Abs�zen nach DIN (1,5 Zeilen)
% \setlength{\parskip}{1.5ex plus0.5ex minus0.5ex}

% Einrckung am Anfang eines neuen Absatzes nach DIN (keine)
%\setlength{\parindent}{0pt}

% R�der definieren
% \setlength{\oddsidemargin}{0.3cm}
% \setlength{\textwidth}{15.6cm}

% bessere Bildunterschriften
%\usepackage[center]{caption2}


% Probleml�ungen beim Umgang mit Gleitumgebungen
\usepackage{float}

% Nummeriert bis zur Strukturstufe 3 (also <section>, <subsection> und <subsubsection>)
%\setcounter{secnumdepth}{3}

% Fhrt das Inhaltsverzeichnis bis zur Strukturstufe 3
%\setcounter{tocdepth}{3}

\usepackage{exscale}

\newenvironment{dsm} {\begin{displaymath}} {\end{displaymath}}
\newenvironment{vars} {\begin{center}\scriptsize} {\normalsize \end{center}}


\newcommand {\en} {\varepsilon_0}               % Epsilon-Null aus der Elektrodynamik
\newcommand {\lap} {\; \mathbf{\Delta}}         % Laplace-Operator
\newcommand {\R} { \mathbb{R} }                 % Menge der reellen Zahlen
\newcommand {\e} { \ \mathbf{e} }               % Eulersche Zahl
\renewcommand {\i} { \mathbf{i} }               % komplexe Zahl i
\newcommand {\N} { \mathbb{N} }                 % Menge der nat. Zahlen
\newcommand {\C} { \mathbb{C} }                 % Menge der kompl. Zahlen
\newcommand {\Z} { \mathbb{Z} }                 % Menge der kompl. Zahlen
\newcommand {\limi}[1]{\lim_{#1 \rightarrow \infty}} % Limes unendlich
\newcommand {\sumi}[1]{\sum_{#1=0}^\infty}
\newcommand {\rot} {\; \mathrm{rot} \,}         % Rotation
\newcommand {\grad} {\; \mathrm{grad} \,}       % Gradient
\newcommand {\dive} {\; \mathrm{div} \,}        % Divergenz
\newcommand {\dx} {\; \mathrm{d} }              % Differential d
\newcommand {\cotanh} {\; \mathrm{cotanh} \,}   %Cotangenshyperbolicus
\newcommand {\asinh} {\; \mathrm{areasinh} \,}  %Area-Sinus-Hyp.
\newcommand {\acosh} {\; \mathrm{areacosh} \,}  %Area-Cosinus-H.
\newcommand {\atanh} {\; \mathrm{areatanh} \,}  %Area Tangens-H.
\newcommand {\acoth} {\; \mathrm{areacoth} \,}  % Area-cotangens
\newcommand {\Sp} {\; \mathrm{Sp} \,}
\newcommand {\mbe} {\stackrel{\text{!}}{=}}     %Must Be Equal
\newcommand{\qed} { \hfill $\square$\\}
\renewcommand{\i} {\imath}
\def\captionsngerman{\def\figurename{\textbf{Abb.}}}

%%%%%%%%%%%%%%%%%%%%%%%%%%%%%%%%%%%%%%%%%%%%%%%%%%%%%%%%%%%%%%%%%%%%%%%%%%%%
% SWITCH FOR PDFLATEX or LATEX
%%%%%%%%%%%%%%%%%%%%%%%%%%%%%%%%%%%%%%%%%%%%%%%%%%%%%%%%%%%%%%%%%%%%%%%%%%%%
%%%
\ifx\pdfoutput\undefined %%%%%%%%%%%%%%%%%%%%%%%%%%%%%%%%%%%%%%%%% LATEX %%%
%%%
\usepackage[dvips]{graphicx}       %%% graphics for dvips
\DeclareGraphicsExtensions{.eps,.ps}   %%% standard extension for included graphics
\usepackage[ps2pdf]{thumbpdf}      %%% thumbnails for ps2pdf
\usepackage[ps2pdf,                %%% hyper-references for ps2pdf
bookmarks=true,%                   %%% generate bookmarks ...
bookmarksnumbered=true,%           %%% ... with numbers
hypertexnames=false,%              %%% needed for correct links to figures !!!
breaklinks=true,%                  %%% breaks lines, but links are very small
linkbordercolor={0 0 1},%          %%% blue frames around links
pdfborder={0 0 112.0}]{hyperref}%  %%% border-width of frames
%                                      will be multiplied with 0.009 by ps2pdf
%
%\hypersetup{ pdfauthor   = {Hannes Franke; Julius Tilly},
%pdftitle    = {x}, pdfsubject  = {Protokoll FP}, pdfkeywords = {V301, Innenwiderstand, Leistungsanpassung},
%pdfcreator  = {LaTeX with hyperref package}, pdfproducer = {dvips
%+ ps2pdf} }
%%%
\else %%%%%%%%%%%%%%%%%%%%%%%%%%%%%%%%%%%%%%%%%%%%%%%%%%%%%%%%%% PDFLATEX %%%
%%%
\usepackage[pdftex]{graphicx}      %%% graphics for pdfLaTeX
\DeclareGraphicsExtensions{.pdf}   %%% standard extension for included graphics
\usepackage[pdftex]{thumbpdf}      %%% thumbnails for pdflatex
\usepackage[pdftex,                %%% hyper-references for pdflatex
bookmarks=true,%                   %%% generate bookmarks ...
bookmarksnumbered=true,%           %%% ... with numbers
hypertexnames=false,%              %%% needed for correct links to figures !!!
breaklinks=true,%                  %%% break links if exceeding a single line
linkbordercolor={0 0 1},
linktocpage]{hyperref} %%% blue frames around links
%                                  %%% pdfborder={0 0 1} is the default
% \hypersetup{
% pdftitle    = {V301 Innenwiderstand und Leistungsanpassung}, 
% pdfsubject  = {Protokoll AP}, 
% pdfkeywords = {V301, Innenwiderstand, Leistungsanpassung},
% pdfsubject  = {Protokoll AP},
% pdfkeywords = {V301, Innenwiderstand, Leistungsanpassung}}
%                                  %%% pdfcreator, pdfproducer,
%                                      and CreationDate are automatically set
%                                      by pdflatex !!!
\pdfadjustspacing=1                %%% force LaTeX-like character spacing
\usepackage{epstopdf}
%
\fi %%%%%%%%%%%%%%%%%%%%%%%%%%%%%%%%%%%%%%%%%%%%%%%%%%% END OF CONDITION %%%
%%%%%%%%%%%%%%%%%%%%%%%%%%%%%%%%%%%%%%%%%%%%%%%%%%%%%%%%%%%%%%%%%%%%%%%%%%%%
% seitliche Tabellen und Abbildungen
%\usepackage{rotating}
\usepackage{ae}
\usepackage{
  array,
  booktabs,
  dcolumn
}
\makeatletter 
  \renewenvironment{figure}[1][] {% 
    \ifthenelse{\equal{#1}{}}{% 
      \@float{figure} 
    }{% 
      \@float{figure}[#1]% 
    }% 
    \centering 
  }{% 
    \end@float 
  } 
  \makeatother 


  \makeatletter 
  \renewenvironment{table}[1][] {% 
    \ifthenelse{\equal{#1}{}}{% 
      \@float{table} 
    }{% 
      \@float{table}[#1]% 
    }% 
    \centering 
  }{% 
    \end@float 
  } 
  \makeatother 
%\usepackage{listings}
%\lstloadlanguages{[Visual]Basic}
%\allowdisplaybreaks[1]
%\usepackage{hycap}
%\usepackage{fancyunits}

% ========================================
%	Angaben für das Titelblatt
% ========================================

\title{Versuch V57 - Rauschen\\				% Titel des Versuchs 
\large TU Dortmund, Fakultät Physik\\ 
\normalsize Fortgeschrittenen-Praktikum}

\author{Jan Adam\\			% Name Praktikumspartner A
{\small \href{jan.adam@tu-dortmund.de}{jan.adam@tu-dortmund.de}}	% Erzeugt interaktiven einen Link
\and						% um einen weiteren Author hinzuzfügen
Felix Wieland\\					% Name Praktikumspartner B
{\small \href{felix.wieland@tu-dortmund.de}{felix.wieland@tu-dortmund.de}}		% Erzeugt interaktiven einen Link
}
\date{09.06.2015}				% Das Datum der Versuchsdurchführung

% ========================================
%	Das Dokument beginnt
% ========================================

\begin{document}

% ========================================
%	Titelblatt erzeugen
% ========================================

\maketitle					% Jetzt wird die Titelseite erzeugt
\thispagestyle{empty} 				% Weder Kopfzeile noch Fußzeile

% ========================================
%	Der Vorspann
% ========================================

%\newpage					% Wenn Verzeichnisse auf einer neuen Seite beginnen sollen
%\pagestyle{empty}				% Weder Kopf- noch Fußzeile für Verzeichnisse

\tableofcontents

%\newpage					% eine neue Seite
%\thispagestyle{empty}				% Weder Kopf- noch Fußzeile für Verzeichnisse
%\listoffigures					% Abbildungsverzeichnis

%\newpage					% eine neue Seite
%\thispagestyle{empty}				% Weder Kopf- noch Fußzeile für Verzeichnisse
%\listoftables					% Tabellenverzeichnis
\newpage					% eine neue Seite


% ========================================
%	Kapitel
% ========================================

%\section{Einleitung}				% Bei Bedarf

\section{Auswertung}
\subsection{Eigenrauschen}
Zunächst muss das Eigenrauschen der Apparatur bei verschiedenen Gain Einstellungen bestimmt werden, da alle folgenden Messungen um diesen Wert korrigiert werden müssen.

Dazu wurde ein \SI{0}{\ohm} Widerstand an den Eingang angeschlossen und anschließend das Rauschspannungsquadrat $U_0^2$ bei allen Nachverstärkungsstufen $(V_N)$ gemessen. Die Ergebnisse der Messungen stehen in den Tabellen \ref{tab:eigenRauschen_norm} und \ref{tab:eigenRauschen_korr}.

\begin{table}[h]
	\begin{subtable}[c]{0.3\textwidth}
		\begin{tabular}{lS}
			{$V_N$} & {$U_0^2$ [\si{\milli\volt}]}\\
			\toprule
			1	& 4.0\\
			2	& 4.0\\
			5	& 4.0\\
			10	& 4.0\\
			20	& 4.0\\
			50	& 4.0\\
			100	& 3.6\\
			200	& 1.0\\
			500	& 12.0\\
			1000& 55.6
		\end{tabular}
		\caption{Normale Schaltung}
		\label{tab:eigenRauschen_norm}
	\end{subtable}%
	\hspace{1cm}%
	\begin{subtable}[c]{0.3\textwidth}
		\begin{tabular}{lS}
			{$V_N$} & {$U_0^2$ [\si{\milli\volt}]}\\
			\toprule
			1	&-11.1 \\
			2	&-10.9 \\
			5	&-11.0 \\
			10	&-11.1 \\
			20	&-11.0 \\
			50	&-10.8 \\
			100	&-10.2 \\
			200	&-9.8 \\
			500	&0.5 \\
			1000	&53.0
		\end{tabular}
		\caption{Korrelationsschaltung}
		\label{tab:eigenRauschen_korr}
	\end{subtable}
	\caption{Eigenrauschen der Apparatur bei den beiden verwendeten Aufbauten.}
	\label{tab:eigenRauschen}
\end{table}

\subsection{Eichung}
Bevor die eigentlichen Messungen durchgeführt werden können, muss die Apparatur zudem geeicht werden.

Ein Sinuswellengenerator wird mit einer Amplitude von \SI{175}{\milli\volt} an den Eingang angeschlossen und durch einen Abschwächer um den Faktor 1000 reduziert. Der Bandpass wird so eingestellt, dass er nur Frequenzen im Bereich von \SIrange{10}{20}{\kilo\hertz} passieren lässt und die Nachverstärkung wird so eingestellt, dass die Spannungen mit genug Nachkommastellen im Millivoltbereich abgelesen werden kann. Die Messwerte stehen in Tabelle \ref{} und eine graphische Darstellung der mittels Gleichung \ref normierten Spannung $\eta(\nu)$ ist in Abbildung \ref zusehen.

\begin{figure}[h]
	\includegraphics[width=\textwidth]{../auswertung/results/sin.pdf}
	\caption{Messwerte aus der Eichmessung. Der Bandpass wurde auf das Intervall \SI{10}{\kilo\hertz} - \SI{20}{\kilo\hertz} eingestellt.}
\end{figure}

Um die Apparaturkonstante
\begin{align}
	\Delta\nu = \int_0^\infty \eta(\nu)\ d\nu
\end{align}
Zu bestimmen, wird die Integralfläche mittels Trapezregel bestimmt.
Damit ergibt sich für A:
\begin{align}
	\Delta\nu = 3870.61 \pm 
\end{align}
Der Fehler wurde mittels des Restgliedes der Trapezregel abgeschätzt:
\begin{align}
	\textbar R(f)\textbar = \frac{b-a}{12}h^2\text{max}\left(f''(x)\right)
\end{align}

Um die Boltzmann-Konstante $k_B$ zu berechnen, wird der Sinusgenerator durch einen Variablen Widerstand ausgetauscht und die Rauschspannung gegen den Widerstand aufgetragen. Mittels linearer Regression, kann dann $k_B$ bestimmt werden.

\begin{figure}
	\includegraphics[width=\textwidth]{../auswertung/results/wid1.pdf}
	\caption{Thermisches Rauschen Widerstand 1 (\SI{0}{\ohm} - \SI{100}{\ohm})}
	\label{fig:thermRauschen1}
\end{figure}

\begin{figure}
	\includegraphics[width=\textwidth]{../auswertung/results/wid2.pdf}
	\caption{Thermisches Rauschen Widerstand 2 (\SI{0}{\ohm} - \SI{1000}{\ohm})}
	\label{fig:thermRauschen2}
\end{figure}

Mittels der Niquist-Beziehung (Gleichung \ref) lässt sich aus den Steigungen der Ausgleichsgeraden die Boltzmannkonstante berechnen.

\begin{align}
	k_B = \frac{m}{4T\Delta\nu}
\end{align}

\vspace{2cm}
\textbf{Literatur}

\vspace{0.3cm}
[1] Anleitung des Lehrstuhlversuchs

% ========================================
%	Literaturverzeichnis
% ========================================

%\bibliographystyle{plainnat}			% Bibliographie-Style auswählen
%\bibliography{BIBDATEI}			% Literaturverzeichnis

% ========================================
%	Das Dokument endent
% ========================================

\end{document}
