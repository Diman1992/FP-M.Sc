% ========================================
%	Header einbinden
% ========================================

\input{longheader.tex}

% ========================================
%	Angaben für das Titelblatt
% ========================================

\title{Lehrstuhlversuch E1 - Röntgenreflektometrie\\				% Titel des Versuchs 
	\large TU Dortmund, Fakultät Physik\\ 
	\normalsize Fortgeschrittenen-Praktikum}

\author{Jan Adam\\			% Name Praktikumspartner A
	{\small \href{jan.adam@tu-dortmund.de}{jan.adam@tu-dortmund.de}}	% Erzeugt interaktiven einen Link
	\and						% um einen weiteren Author hinzuzfügen
	Felix Wieland\\					% Name Praktikumspartner B
	{\small \href{felix.wieland@tu-dortmund.de}{felix.wieland@tu-dortmund.de}}		% Erzeugt interaktiven einen Link
}
\date{25.04.2016}				% Das Datum der Versuchsdurchführung

% ========================================
%	Das Dokument beginnt
% ========================================

\begin{document}
	
% ========================================
%	Titelblatt erzeugen
% ========================================

\maketitle					% Jetzt wird die Titelseite erzeugt
\thispagestyle{empty} 				% Weder Kopfzeile noch Fußzeile

% ========================================
%	Der Vorspann
% ========================================

%\newpage					% Wenn Verzeichnisse auf einer neuen Seite beginnen sollen
%\pagestyle{empty}				% Weder Kopf- noch Fußzeile für Verzeichnisse

\tableofcontents

%\newpage					% eine neue Seite
%\thispagestyle{empty}				% Weder Kopf- noch Fußzeile für Verzeichnisse
%\listoffigures					% Abbildungsverzeichnis

%\newpage					% eine neue Seite
%\thispagestyle{empty}				% Weder Kopf- noch Fußzeile für Verzeichnisse
%\listoftables					% Tabellenverzeichnis
\newpage					% eine neue Seite


% ========================================
%	Kapitel
% ========================================

\section{Auswertung}
\subsection{Eigenrauschen}
Zunächst muss das Eigenrauschen der Apparatur bei verschiedenen Gain Einstellungen bestimmt werden, da alle folgenden Messungen um diesen Wert korrigiert werden müssen.

Dazu wurde ein \SI{0}{\ohm} Widerstand an den Eingang angeschlossen und anschließend das Rauschspannungsquadrat $U_0^2$ bei allen Nachverstärkungsstufen $(V_N)$ gemessen. Die Ergebnisse der Messungen stehen in den Tabellen \ref{tab:eigenRauschen_norm} und \ref{tab:eigenRauschen_korr}.

\begin{table}[h]
	\centering
	\begin{subtable}[c]{0.3\textwidth}
		\begin{tabular}{lS}
			{$V_N$} & {$U_0^2$ [\si{\milli\volt}]}\\
			\toprule
			1	& 4.0\\
			2	& 4.0\\
			5	& 4.0\\
			10	& 4.0\\
			20	& 4.0\\
			50	& 4.0\\
			100	& 3.6\\
			200	& 1.0\\
			500	& 12.0\\
			1000& 55.6
		\end{tabular}
		\caption{Normale Schaltung}
		\label{tab:eigenRauschen_norm}
	\end{subtable}%
	\hspace{1cm}%
	\begin{subtable}[c]{0.3\textwidth}
		\begin{tabular}{lS}
			{$V_N$} & {$U_0^2$ [\si{\milli\volt}]}\\
			\toprule
			1	&-11.1 \\
			2	&-10.9 \\
			5	&-11.0 \\
			10	&-11.1 \\
			20	&-11.0 \\
			50	&-10.8 \\
			100	&-10.2 \\
			200	&-9.8 \\
			500	&0.5 \\
			1000	&53.0
		\end{tabular}
		\caption{Korrelationsschaltung}
		\label{tab:eigenRauschen_korr}
	\end{subtable}
	\caption{Eigenrauschen der Apparatur bei den beiden verwendeten Aufbauten.}
	\label{tab:eigenRauschen}
\end{table}

\subsection{Eichung}
Bevor die eigentlichen Messungen durchgeführt werden können, muss die Apparatur zudem geeicht werden.

Ein Sinuswellengenerator wird mit einer Amplitude von \SI{175}{\milli\volt} an den Eingang angeschlossen und durch einen Abschwächer um den Faktor 1000 reduziert. Der Bandpass wird so eingestellt, dass er nur Frequenzen im Bereich von \SIrange{10}{20}{\kilo\hertz} passieren lässt und die Nachverstärkung wird so eingestellt, dass die Spannungen mit genug Nachkommastellen im Millivoltbereich abgelesen werden kann. Die Messwerte stehen in Tabelle \ref{} und eine graphische Darstellung der mittels Gleichung \ref normierten Spannung $\eta(\nu)$ ist in Abbildung \ref{fig:eichmessung} zusehen.

\begin{figure}[h]
	\includegraphics[height=8.5cm]{../auswertung/results/sin.pdf}
	\caption{Messwerte aus der Eichmessung. Der Bandpass wurde auf das Intervall \SI{10}{\kilo\hertz} - \SI{20}{\kilo\hertz} eingestellt.}
	\label{fig:eichmessung}
\end{figure}


Um die Apparaturkonstante
\begin{align}
	\Delta\nu = \int_0^\infty \eta(\nu)\ d\nu
\end{align}
Zu bestimmen, wird die Integralfläche mittels Trapezregel bestimmt.
Damit ergibt sich für die normale Schaltung:
\begin{align}
\Delta\nu = (\input{../auswertung/results/sin1_DeltaNu.fit}) \SI{}{\hertz}
\end{align}

Die Fehler wurde mittels des Restgliedes der Trapezregel abgeschätzt:
\begin{align}
	\textbar R(f)\textbar = \frac{b-a}{12}h^2\text{max}\left(f''(x)\right)
	\label{eq:restglied}
\end{align}
Mit diesem Wert kann nun die Rauschzahl der Schaltung berechnet werden. 
\begin{align*}
	T &= \SI{293}{\kelvin}\\
	V^2_\text{ges} &= V_=V^2_VV^2_N = 10^{13}\, \si{\volt}\\
	U^2(500\Omega) &= (706.0\pm 14.1)\,{\milli\volt}\\
	F(500\Omega, \Delta\nu) &= \frac{U^2(500)}{4k_\text{B} \cdot \SI{293}{\kelvin} \cdot \SI{500}{\ohm} \cdot \Delta\nu \cdot V^2_\text{ges}} = \input{../auswertung/results/rausch1.fit}
\end{align*}
Der Fehler für die Spannung $U^2$ wurde dazu mit $2\%$ abgeschätzt.\\

Um die Boltzmann-Konstante $k_B$ zu berechnen, wird der Sinusgenerator durch einen Variablen Widerstand ausgetauscht und die Rauschspannung gegen den Widerstand aufgetragen. Mittels linearer Regression, kann dann über die Nyquist-Beziehung (Gleichung \ref) $k_B$ bestimmt werden. Die graphische Darstellung der Messwerte ist in den Abbildungen \ref{fig:thermRauschen1} und \ref{fig:thermRauschen2} zu finden.\\

Ergebnis der Regression mit dem \SI{1000}{\ohm} Widerstand.
\begin{align*}
	m &= \input{../auswertung/results/wid1_m.fit}\\
	b &= \input{../auswertung/results/wid1_b.fit}\\
	k_{B,1000\,\Omega} &= \frac{m}{4T\Delta\nu} =\input{../auswertung/results/wid1_kb.fit}\\
\end{align*}

Ergebnis der Regression mit dem \SI{100}{\ohm} Widerstand.
\begin{align*}
m &= \input{../auswertung/results/wid2_m.fit}\\
b &= \input{../auswertung/results/wid2_b.fit}\\
k_{B,100\,\Omega} &= \input{../auswertung/results/wid2_kb.fit}
\end{align*}

\begin{figure}[H]
	\includegraphics[width=\textwidth]{../auswertung/results/wid1.pdf}
	\caption{Thermisches Rauschen mit dem \SI{1000}{\ohm} Widerstand.}
	\label{fig:thermRauschen1}
\end{figure}

\begin{figure}[H]
	\includegraphics[width=\textwidth]{../auswertung/results/wid2.pdf}
	\caption{Thermisches Rauschen mit dem \SI{100}{\ohm} Widerstand.}
	\label{fig:thermRauschen2}
\end{figure}


\subsection{Korrelator-Schaltung}
Nun wird die gleiche Rechnung für die Korrelatorschaltung wiederholt. Die Verstärkungsfaktoren sind dabei bis auf einen zusätzlichen Faktor 10 durch den Selektivverstärker identisch zur vorherigen Schaltung.

Begonnen wird mit der Eichung. Mittels Trapezregel werden die Datenpunkte aus Abbildung \ref{fig:eichmessung_korrel} integriert. Der Fehler übersteigt den Messwert um mehrere Größenordnungen, da für die Restgliedabschätzung (Gleichung \ref{eq:restglied}) die zweite Ableitung verwendet wird und im Bereich des Peaks zu wenige Messungen durchgeführt wurden. Für alle folgenden Rechnungen wird daher der Fehler ignoriert.
\begin{align}
\Delta\nu= (\input{../auswertung/results/sin2_DeltaNu.fit})\,\si{\hertz}
\end{align}

Auch von der Korrelatorschaltung wird die Rauschzahl nach Gleichung \ref bestimmt

\begin{align*}
T &= \SI{293}{\kelvin}\\
V^2_\text{ges} &= V_=V^2_VV^2_NV_S^2 = 10^{15}\, \si{\volt}\\
U^2(500\Omega) &= (625.0\pm 12.5)\,{\milli\volt}\\
F(500\Omega, \Delta\nu) &= \input{../auswertung/results/rausch2.fit}
\end{align*}

\begin{figure}[htbp]
	\centering
	\includegraphics[height=8.5cm]{../auswertung/results/sin_korrel.pdf}
	\caption{Messwerte aus der Eichmessung der Korrelationsschaltung. Die Selektivverstärker wurden auf \SI{15}{\kilo\Hz} eingeregelt.}
	\label{fig:eichmessung_korrel}
\end{figure}

und mittels Nyquist-Beziehung die Boltzmann Konstante $k_\text{B}$ ermittelt.

Bei der Korrelatorschaltung lässt sich die Rauschspannung zudem durch folgende Formel ausdrücken:
\begin{align}
	U^2_a = \frac{U_e^2}{Q^2}\frac{1}{\eta^2 + \eta^{-2} + Q^{-2} -2} \qquad \qquad ;\eta = \frac{\nu}{\nu_0}
\end{align}
Mit dem least-square Verfahren wurde die Formel an die Messpunkte angenähert (Abbildung \ref{fig:eichmessung_korrel}). Bei der grünen Kurve wurden alle drei Parameter durch den Algorithmus ermittelt und bei der roten Kurve wurden die Parameter $Q=10$ und $V_E=0.175\,\si{mV}$ gesetzt, was der Konfiguration des Experimentes entspricht.
\begin{align}
	\nu_0 &= (\input{../auswertung/results/sin2_nu0.fit}) \,\hertz\\
	Q &= (\input{../auswertung/results/sin2_Q.fit})\\
	V_E &= \input{../auswertung/results/sin2_U.fit}\, \si{\volt}
\end{align}

Der Sinusgenerator wird durch die Widerstände ersetzt:\\
Ergebnis der Regression mit dem \SI{1000}{\ohm} Widerstand.
\begin{align*}
m &= \input{../auswertung/results/wid1_korrel_m.fit}\\
b &= \input{../auswertung/results/wid1_korrel_b.fit}\\
k_{B,1000\,\Omega} &= \frac{m}{4T\Delta\nu} =\input{../auswertung/results/wid1_korrel_kb.fit}\\
\end{align*}

Ergebnis der Regression mit dem \SI{100}{\ohm} Widerstand.
\begin{align*}
m &= \input{../auswertung/results/wid2_korrel_m.fit}\\
b &= \input{../auswertung/results/wid2_korrel_b.fit}\\
k_{B,100\,\Omega} &= \input{../auswertung/results/wid2_korrel_kb.fit}
\end{align*}

\begin{figure}[H]
	\includegraphics[width=\textwidth]{../auswertung/results/wid1_korrel.pdf}
	\caption{Thermisches Rauschen mit dem \SI{1000}{\ohm} Widerstand.}
	\label{fig:thermRauschen1}
\end{figure}

\begin{figure}[H]
	\includegraphics[width=\textwidth]{../auswertung/results/wid2_korrel.pdf}
	\caption{Thermisches Rauschen mit dem \SI{100}{\ohm} Widerstand.}
	\label{fig:thermRauschen2}
\end{figure}



\subsection{Reinmetall-Kathode}
Zunächst werden Kennlinien der Reinmetall-Kathode aufgenommen, damit sichergestellt werden kann, dass die Kathode bei der Datennahme im Sättigungsbereich betrieben wird. Die Messwerte sind in Abbildung \ref{fig:kennlinie} dargestellt. Für alle folgenden Messungen wird eine Spannung von \SI{150}{\volt} eingestellt.
\begin{figure}[h]
	\includegraphics[width=0.8\textwidth]{../auswertung/results/kathRein_sat.pdf}
	\caption{Kennlinien der Reinkathode aufgenommen bei zwei verschiedenen Heizströmen.}
	\label{fig:kennlinie}
\end{figure}

Die Schottky-Beziehung kann zu
\begin{align}
	U_a^2 = \frac{2e_0 \ \Delta\nu}{R} I_0
\end{align}



\vspace{2cm}
\textbf{Literatur}

\vspace{0.3cm}
[1] Anleitung des Lehrstuhlversuchs

% ========================================
%	Literaturverzeichnis
% ========================================

%\bibliographystyle{plainnat}			% Bibliographie-Style auswählen
%\bibliography{BIBDATEI}			% Literaturverzeichnis

% ========================================
%	Das Dokument endent
% ========================================
\end{document}
