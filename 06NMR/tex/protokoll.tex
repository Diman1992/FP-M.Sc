% ========================================
%	Header einbinden
% ========================================

\input{longheader.tex}
\usepackage{xfrac}
\usepackage{xcolor}
\usepackage{setspace}\usepackage{threeparttable}
\usepackage{fancyhdr}
\usepackage{graphicx}
\usepackage[official]{eurosym}
\usepackage{geometry}

% ========================================
%	Angaben für das Titelblatt
% ========================================

\title{V49 - Messung von Diffusionskonstanten mittels gepulster Kernspinresonanz\\
\hspace{15cm}% Titel des Versuchs 
	\large TU Dortmund, Fakultät Physik\\ 
	\normalsize Fortgeschrittenen-Praktikum}

\author{Dimitrios Skodras\\			% Name Praktikumspartner A
	{\small \href{dimitrios.skodras@tu-dortmund.de}{dimitrios.skodras@tu-dortmund.de}}	% Erzeugt interaktiven einen Link
	\and						% um einen weiteren Author hinzuzfügen
	Julius Tilly\\					% Name Praktikumspartner B
	{\small \href{julius.tilly@tu-dortmund.de}{julius.tilly@tu-dortmund.de}}		% Erzeugt interaktiven einen Link
}
\date{12.12.2016}				% Das Datum der Versuchsdurchführung

% ========================================
%	Das Dokument beginnt
% ========================================

\begin{document}
	
% ========================================
%	Titelblatt erzeugen
% ========================================

\maketitle					% Jetzt wird die Titelseite erzeugt
\thispagestyle{empty} 				% Weder Kopfzeile noch Fußzeile

% ========================================
%	Der Vorspann
% ========================================

%\newpage					% Wenn Verzeichnisse auf einer neuen Seite beginnen sollen
%\pagestyle{empty}				% Weder Kopf- noch Fußzeile für Verzeichnisse

\tableofcontents

%\newpage					% eine neue Seite
%\thispagestyle{empty}				% Weder Kopf- noch Fußzeile für Verzeichnisse
%\listoffigures					% Abbildungsverzeichnis

%\newpage					% eine neue Seite
%\thispagestyle{empty}				% Weder Kopf- noch Fußzeile für Verzeichnisse
%\listoftables					% Tabellenverzeichnis
\newpage					% eine neue Seite


% ========================================
%	Kapitel
% ========================================

\section{Theoretische Grundlagen}
Kernspinresonanz beschreibt die Messung der makroskopischen Magnetisierung, erzeugt durch ein äußeres Magnetfeld, abhängig vom Kernspin der
untersuchten Probe.
\subsection{Kernspinresonanz}
Dieses Magnetfeld $\vec B=B_0\vec z$ verursacht die Aufspaltung der entarteten Kernspinzustände in $2I+1$ Unterniveaus mit je 
$\Delta E = \gamma B_0 \hbar$ zueinander. Hierbei sind $I$ die Kernspinquantenzahl und $\gamma$ das gyromagnetische Moment. Für Wasserstoff ist
$I=\sfrac12$, also gibt es zwei Unterniveaus mit den Orientierungsquantenzahlen $m=\pm\sfrac12$, die nach der Boltzmann-Verteilung ungleich besetzt 
sind $\sim\exp\left(-\Delta E/kT\right)$.
Dadurch entsteht eine Kernspinpolarisation $\langle I_z \rangle$, die mit $\Delta E \ll kT$ berechnet werden kann. 
Die einzelnen magnetischen Momente $\vec \mu_I$ der einzelnen Kerne werden nun zur makroskopischen Magnetisierung $\vec M_0$ der Probe aufsummiert,
mit ihrem Erwartungswert in $z$-Richtung $\langle M_0\rangle = N \gamma \mu_0 \langle I_z\rangle$. Hierbei ist $N$ die Anzahl der Momente pro Volumen.
Gemessen wird die Magnetisierung durch das Auslösen ihrer Ruhelage mittels eines Hochfrequenzpulses (HF). Diese präzediert um das äußere Magnetfeld,
ausgedrückt durch die Differenzialgleichung
\begin{align}
 \frac{\dx}{\dx t}\vec M = \gamma \vec M \times B_0 \vec z.
 \label{eq:Mdgl}
\end{align}
Ihre Komponenten $M_x$ und $M_y$ beschreiben eine Kreisbewegung mit der Frequenz $\omega_L := \gamma B_0$, der Larmor-Frequenz. Nachdem
der HF-Puls die Magnetisierung gestört hat, strebt sie wieder ihrem Ursprungszustand $\vec M_0$ entgegen, sie relaxiert. Die Relaxation
und die Präzession zusammengefasst ergeben die Bloch-Gleichungen
\begin{align}
 \frac{\dx}{\dx t}M_x &= \omega_L M_y - \frac{M_x}{T_2},\\
 \frac{\dx}{\dx t}M_y &= \omega_L M_x - \frac{M_y}{T_2},\\
 \frac{\dx}{\dx t}M_z &= \frac{M_0-M_z}{T_1},
\end{align}
mit den Relaxationszeiten $T_1$ und $T_2$. Diese beschreiben den Energieübergang aus dem Spinsystem ins Gitter, bzw. die Wechselwirkung
benachbarter Spins. \\
\noindent Die genannten HF-Pulse werden von einem Signalgenerator bei der Larmorfrequenz erzeugt und für eine bestimmte Pulslänge 
$\Delta t_\delta$ an eine Spule geleitet, die ein Magnetfeld $\vec B_\text{HF} = 2\vec B_1 \cos(\omega t)$ generiert. Dieses ist senkrecht
zu und präzediert um $\vec z$ und kann zusammen mit $B_0$ als $B_\text{ges}$ geschrieben werden. In einem mit $\omega$ um $\vec z$ 
rotierenden Koordinatensystem $\{\vec x', \vec y', \vec z\}$ ist das $B_1$-Feld zeitlich konstant, welches o.B.d.A in $\vec x'$ Richtung zeigt. 
Da nun $\vec x' = \vec x'(t)$, ergibt sich für 
\eqref{eq:Mdgl}
\begin{align}
 \frac{\dx }{\dx t}\vec M = \gamma \vec M  \times \left(\vec B_\text{ges} + \frac{\vec \omega}{\gamma}\right).
\end{align}
Hiermit lässt sich ein $\vec B_\text{eff} = \vec B_\text{ges} + \vec\omega/\gamma$ schreiben, um das sich $\vec M$ effektiv aus $\vec z$ herausdreht. Wenn
die eingestrahlte Frequenz gleich der Larmorfrequenz ist, ist $\vec \omega$ antiparallel zu $\vec z$ und folglich ist $\vec B_\text{eff}=\vec B_1$.
Die Dauer in der das Magnetfeld die Magnetisierung um den Winkel $\delta$ aus der $\vec z$-Achse dreht, ergibt sich nun zu
$\Delta t_\delta = \delta/\gamma B_1$.

\subsection{Messmethoden von den Relaxationszeiten und der Diffusionskonstanten}
Zur Bestimmung der Relaxationszeiten $T_1$ und $T_2$ wird die Magnetisierung, wie bereits beschrieben, in die $x'$-$y'$-Ebene mit einem 
$\sfrac{\pi}{2}$-Puls gebracht. Die Präzession der Magnetisierung induziert eine Spannung in der HF-Spule, die gemessen wird.
\subsubsection{Spin-Gitter-Relaxation}
Beginnend mit einem $\pi$-Puls wird die Gleichgewichtsmagnetisierung $M_0$ in $-\vec z$-Richtung gebracht. Diese geht nun wieder mit der Zeit 
in $+\vec z$-Richtung. Nach einer gewissen Wartezeit $\tau$ wird ein $\sfrac{\pi}{2}$-Puls erzeugt, womit eine Spannung proportional
zu
\begin{align}
 M_z(\tau) = M_0\left(1-2\exp\left(\frac{\tau}{T_1}\right)\right)
\end{align}
induziert wird. Mit Messungen für diverse $\tau$ lässt sich $T_1$, die Spin-Gitter-Relaxationszeit, bestimmen.

\subsubsection{Spin-Spin-Relaxation}
Ähnlich simpel ist die Beschreibung der Spin-Spin-Relaxationszeit $T_2$ mit dem freien Induktionszerfall. Hierzu wird ein $\sfrac{\pi}{2}$-Puls
durchgeführt, woraufhin die Magnetisierung wieder zur Ruhelage tendiert. Die Induktionsspannung im Laufe der Zeit liefert die Bestimmung von $T_2$.
Durch Magnetfeldinhomogenitäten und Wechselwirkungen mit benachbarten Dipolfeldern jedoch, gibt es eine Verteilung von Larmorfrequenzen, was 
schneller bzw. langsamer präzedierenden Momenten enspricht. Diese Dephasierung lässt sich ausdrücken durch $T_{\Delta B}$ und beeinflusst
die $T_2$-Messung maßgeblich, wenn $T_{\Delta B} < T_2$. Diese Konstante wird beschrieben durch $T_{\Delta B} = 1/\gamma d G$ mit dem 
Probendurchmesser $d$ und $G$, dem $B_0$-Gradienten.\\
\noindent Eine erste Verbesserung dazu ist die Spin-Echo-Methode. Hierzu wird der Dephasierung nach einer Zeit $\tau $ mit einem $\pi$-Puls 
entgegengewirkt, sodass die Momente wieder zusammenlaufen. Nach $2\tau$ wird eine Resonanz gemessen. Durch irreversible Dephasierungsprozesse
ist die gemessene Magnetisierung abhängig von $\tau$, ausgedrückt durch
\begin{align}
 M_y(\tau) = M_0\exp\left(-\frac{\tau}{T_2} \right).
\end{align}
Dieses Verfahren ist langwierig, da gewartet werden muss, bist die Magnetisierung wieder vollständig relaxiert ist. Die Carr-Purcell-Methode (CP)
fügt viele weitere $\pi$-Pulse bei $2n\tau$ ($n\in\mathbb{N}$) hinzu, wobei die Resonanzamplitude mit wachsendem $n$ abnimmt. Eine Fehlerquelle
hierin ist die Bedingung für exakt bestimmte $\Delta t_{\pi}$, denn anderweitig sind liegen die Momente um einen Winkel $\epsilon$ aus der 
$x'$-$y'$-Ebene heraus. Mit jedem $\pi$-Puls wird $\epsilon$ weiter addiert, sodass $T_2$ zu klein gemessen wird. Eine weitere Methode, die 
diesen Nachteil beseitigt, ist die Meiboom-Gill-Methode (MG). Ähnlich der CP-Methode werden die $\pi$-Pulse jedoch um $\pi/2$ gegen den 
$\pi/2$-Puls phasenverschoben, sodass $\vec B$ direkt zu Beginn in $\vec y'$-Richtung zeigt. Der Überschüssige Winkel $\epsilon$ wird beim
zweiten $\pi$-Puls wieder aufgehoben. Dieses Verfahren ist in Abbildung \ref{pic:MGmeth} dargestellt.
\begin{figure}[t]
 \includegraphics[width=.7\textwidth]{../pics/MG.png}
 \caption{Darstellung der Meiboom-Gill-Methode}
 \label{pic:MGmeth}
\end{figure}

\subsubsection{Einfluss der Diffusion}


\section{Durchführung}
\section{Auswertung}
\section{Diskussion}



 \begin{thebibliography}{WissOnl}
 	\bibitem{Anl} TU Dortmund instruction for experiment Nr.60 \url{http://129.217.224.2/HOMEPAGE/Anleitung_FPBSc.html}
 	\bibitem{ref} Laser spectroscopy of Rubidium \\ \url{http://www.photodigm.com/literature/applications-notes/rubidium-absorption-spectroscopy}
 \end{thebibliography}

% ========================================
%	Literaturverzeichnis
% ========================================

%\bibliographystyle{plainnat}			% Bibliographie-Style auswählen
%\bibliography{BIBDATEI}			% Literaturverzeichnis

% ========================================
%	Das Dokument endent
% ========================================
\end{document}
