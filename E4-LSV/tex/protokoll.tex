% ========================================
%	Header einbinden
% ========================================

\input{longheader.tex}
\usepackage{subcaption}
\usepackage{placeins}

% ========================================
%	Angaben für das Titelblatt
% ========================================

\title{Search for $t\bar t$ Resonances with ATLAS data \\ \vspace{.5cm}% Titel des Versuchs 
	\large TU Dortmund, Fakultät Physik\\ 
	\normalsize Fortgeschrittenen-Praktikum}

\author{Dimitrios Skodras\\			% Name Praktikumspartner A
	{\small \href{dimitrios.skodras@tu-dortmund.de}{dimitrios.skodras@tu-dortmund.de}}	% Erzeugt interaktiven einen Link
				\and			% um einen weiteren Author hinzuzfügen
	Julius Tilly\\					% Name Praktikumspartner B
	{\small \href{julius.tilly@tu-dortmund.de}{julius.tilly@tu-dortmund.de}}		% Erzeugt interaktiven einen Link
}
\date{27.09.2016}				% Das Datum der Versuchsdurchführung

% ========================================
%	Das Dokument beginnt
% ========================================

\begin{document}
	
% ========================================
%	Titelblatt erzeugen
% ========================================

\maketitle					% Jetzt wird die Titelseite erzeugt
\thispagestyle{empty} 				% Weder Kopfzeile noch Fußzeile

% ========================================
%	Der Vorspann
% ========================================

%\newpage					% Wenn Verzeichnisse auf einer neuen Seite beginnen sollen
%\pagestyle{empty}				% Weder Kopf- noch Fußzeile für Verzeichnisse

\tableofcontents

%\newpage					% eine neue Seite
%\thispagestyle{empty}				% Weder Kopf- noch Fußzeile für Verzeichnisse
%\listoffigures					% Abbildungsverzeichnis

%\newpage					% eine neue Seite
%\thispagestyle{empty}				% Weder Kopf- noch Fußzeile für Verzeichnisse
%\listoftables					% Tabellenverzeichnis
\newpage					% eine neue Seite


% ========================================
%	Kapitel
% ========================================

\section{Introduction}
Besides its properties to have very accurate estimations for the processes it is applicable to, the Standard Model (SM) lacks the explanation for 
several phenomena. Many of them hint at the electroweak scale of order $\mathcal{O}$(100 GeV) where new physics is expected to occur. One new particle
proposed by some models beyond the SM is $Z'$. Its properties are very similar to the SM-$Z$ boson but is around an order of magnitude heavier so that
it can decay into a top-antitop pair $t\bar t$. It may furthermore serve as a mediator to a dark sector or may have flavour violatioting interactions, 
possibly 
explaining flavour anomalies in $B$-meson physics. Considerations as the latter two, however, are beyond the scope of this work. The top pair mainly
gets produced via the strong interaction of two gluons $g$ and decays weakly into two $b$-quarks and two $W$-bosons which themselves decay further 
expressed by three topologies. One may decay into a lepton pair and the other into a quark pair (i.e. lepton$+$jets) or both decay leptonically 
(i.e. dilepton) or hadronically (i.e. all hadron). Due to its color representation and the quark flavour violation, decays into quarks is much more 
probable compared tolepton involving ones. But they are more complicate to identify, i.e. reconstructed, as $t$ decay products as a result of 
confinement. The tradeoff is therefore to focus on the lepton$+$jets topology, while we do not consider $\tau$ because of its various decay channels.
\\
\noindent The data \cite{Atlas} of the preselected sets gets further sorted out by additional requirements to obtain the requested topology. From direct and 
derived kinematic entities one final discriminant, the invariant mass of the system, gets picked to separate the background from the signal emerging
as a sharp peak. Background processes are modeled by Monte Carlo (MC) simulations whose validity is reviewed in the process. In the end, a statistical
analysis shows no significance for neither an observation nor evidence for a $Z'$ resonance but a lower mass is set at $m_{Z'} >850$ GeV at 
95\% CL


\FloatBarrier
\section{Selection conditions and their efficiencies}
An example for a depiction of a dataset is shown in figure \ref{pic:examplePlot}. It represents a histogram fro the mass of the jets (jet\_m) in an
event. Since mostly two jets are reconstructed $b$-quarks, i.e. b-tagged, the peak is a bit below twice the $b$-quark mass. 
\begin{figure}[t]
 \includegraphics[width= 0.5\textwidth]{../pics/2/jetm.pdf}
 \caption{Histogram of the jet mass from data set 0.}
 \label{pic:examplePlot}
\end{figure}
%TODO: maybe one sentence about the difference between #_plot and #_ntuple (2.1b)
The used samples are already preselected with certain requirements. These include a lower bound on the momentum transverse to the beam axis $p_T>17$ GeV 
\footnote[1]{Leptons coming from electroweak
decays are rare in general but have a quite high $p_T$ expressed by this lower bound}and
for the pseudorapidity $\eta$ which represents the alignment of the jet with the beam, we demand $|\eta| < 2.47$. With the detector being rotationally
invariant along the beam axis, no requirement for the azimuthal angle $\phi$ is posed. To cover the requested topology, we demand 
one lepton to be reconstructed. 

2+3
% Notes:- plot description of example jet_m. 
% 	- #_plot > #_ntuple? holds for leptonic entities: difference in number represents reconstructed leptons
% 	- trigger and reconstruction requirements are maybe found in Jana_Novakova book ore elsewhere in the literature
% 	- pT>17GeV (otherwise too much background) and eta<2,47 (interested in ``central'' letpons), phi has no constraints due to rotational invariance of the detector
% 	- 3.1a not sure what they mean, but maybe: all-hadronic: bbqq; lepton+jets: bbql; dilepton: bbll. b=b-quark, q=any lighter quark, l=lepton.
% 	- 3.1b if-environments in runSelection.C
% 	- 3.1c ttbar, diboson, ww, zz, singletop - topology: bbql for amount of events and detectability. 
% 	- 3.1e maybe include the table in output.info.txt
% 	- 3.1g efficiencies from outputRunselection/; requirements: jetGood>lepN>met>btagged>jetN (with rising efficiency) for more or less all background processes
% 	- efficiency for Z'(1000) is 21% 
% 	- 3.1h main selected background process is ttbar with 10e5 events compared to the others being of order 10e3 or less
	

\section{Fundamental and derived quantities}

\newpage
\begin{figure}
\begin{minipage}{0.32\textwidth}
 \includegraphics[width=\textwidth]{../pics/4/leppt.pdf}
\end{minipage}
\begin{minipage}{0.32\textwidth}
 \includegraphics[width=\textwidth]{../pics/4/jetpt.pdf}
\end{minipage}
\begin{minipage}{0.32\textwidth}
 \includegraphics[width=\textwidth]{../pics/4/jetptLarge.pdf}
\end{minipage}

\begin{minipage}{0.33\textwidth}
 \includegraphics[width=\textwidth]{../pics/4/lepeta.pdf}
\end{minipage}
\begin{minipage}{0.33\textwidth}
 \includegraphics[width=\textwidth]{../pics/4/jeteta.pdf}
\end{minipage}
\begin{minipage}{0.33\textwidth}
 \includegraphics[width=\textwidth]{../pics/4/jetetaLarge.pdf}
\end{minipage}

\begin{minipage}{0.33\textwidth}
 \includegraphics[width=\textwidth]{../pics/4/lepphi.pdf}
\end{minipage}
\begin{minipage}{0.33\textwidth}
 \includegraphics[width=\textwidth]{../pics/4/jetphi.pdf}
\end{minipage}
\begin{minipage}{0.33\textwidth}
 \includegraphics[width=\textwidth]{../pics/4/jetphiLarge.pdf}
\end{minipage}

\begin{minipage}{0.33\textwidth}
 \includegraphics[width=\textwidth]{../pics/4/lepE.pdf}
\end{minipage}
\begin{minipage}{0.33\textwidth}
 \includegraphics[width=\textwidth]{../pics/4/jetE.pdf}
\end{minipage}
\begin{minipage}{0.33\textwidth}
 \includegraphics[width=\textwidth]{../pics/4/jetELarge.pdf}
\end{minipage}

\begin{minipage}{0.33\textwidth}
 \includegraphics[width=\textwidth]{../pics/4/jetnum.pdf}
\end{minipage}
\begin{minipage}{0.33\textwidth}
 \includegraphics[width=\textwidth]{../pics/4/jetbtag.pdf}
\end{minipage}
\begin{minipage}{0.33\textwidth}
 \includegraphics[width=\textwidth]{../pics/4/misstrans.pdf}
\end{minipage}
\caption{bla}
\label{pic:bla}
\end{figure}




4+5
% Notes:- 4a+b: comparing a data set with ttbar
% 	- taking inv_massSys as most powerful discriminant because it has resonances (peaks) at the particles or systems mass which is known (ttbar) or seeked (Z')

\section{Data-MC comparisons and constraints on $Z'$-mass}
6+7
% Notes:- 6.2c: some have fluctuations -> nothing to worry about. But in jet_eta, jet_pt and lep_eta for example the data points are almost always below the MC-expectation by fairly the same amount ->maybe there is a small, overall factor missing
% 	- 6.2d: from inv_massSys no Z' signal can be seen by eye.
% 	- 7.1a: from results/notes we get a p-value of 0.46. For 95 C.F. you get chi=36.4149 which means that the impact of the Z' cross section gets increased until this chi is reached.
% 	- 7.1d: by doing this, the lighter Z' overproduce the observed jets up to the mass of 850 GeV above which the Z' boson is still viable

\section{Conclusion}

 \begin{thebibliography}{WissOnl}
 	\bibitem{Atlas} ATLAS Collaboration, The ATLAS Experiment at the CERN Large Hadron Collider, JINST 3 (2008) S08003
 	\bibitem{ref} Laser spectroscopy of Rubidium \\ \url{http://www.photodigm.com/literature/applications-notes/rubidium-absorption-spectroscopy}
 \end{thebibliography}

% ========================================
%	Literaturverzeichnis
% ========================================

%\bibliographystyle{plainnat}			% Bibliographie-Style auswählen
%\bibliography{BIBDATEI}			% Literaturverzeichnis

% ========================================
%	Das Dokument endent
% ========================================
\end{document}
