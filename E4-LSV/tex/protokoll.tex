% ========================================
%	Header einbinden
% ========================================

\input{longheader.tex}

% ========================================
%	Angaben für das Titelblatt
% ========================================

\title{Search for $t\bar t$ Resonances with ATLAS data \\ \vspace{.5cm}% Titel des Versuchs 
	\large TU Dortmund, Fakultät Physik\\ 
	\normalsize Fortgeschrittenen-Praktikum}

\author{Dimitrios Skodras\\			% Name Praktikumspartner A
	{\small \href{dimitrios.skodras@tu-dortmund.de}{dimitrios.skodras@tu-dortmund.de}}	% Erzeugt interaktiven einen Link
				\and			% um einen weiteren Author hinzuzfügen
	Julius Tilly\\					% Name Praktikumspartner B
	{\small \href{julius.tilly@tu-dortmund.de}{julius.tilly@tu-dortmund.de}}		% Erzeugt interaktiven einen Link
}
\date{27.09.2016}				% Das Datum der Versuchsdurchführung

% ========================================
%	Das Dokument beginnt
% ========================================

\begin{document}
	
% ========================================
%	Titelblatt erzeugen
% ========================================

\maketitle					% Jetzt wird die Titelseite erzeugt
\thispagestyle{empty} 				% Weder Kopfzeile noch Fußzeile

% ========================================
%	Der Vorspann
% ========================================

%\newpage					% Wenn Verzeichnisse auf einer neuen Seite beginnen sollen
%\pagestyle{empty}				% Weder Kopf- noch Fußzeile für Verzeichnisse

\tableofcontents

%\newpage					% eine neue Seite
%\thispagestyle{empty}				% Weder Kopf- noch Fußzeile für Verzeichnisse
%\listoffigures					% Abbildungsverzeichnis

%\newpage					% eine neue Seite
%\thispagestyle{empty}				% Weder Kopf- noch Fußzeile für Verzeichnisse
%\listoftables					% Tabellenverzeichnis
\newpage					% eine neue Seite


% ========================================
%	Kapitel
% ========================================

\section{Introduction}


\section{Selection conditions and their efficiencies}
2+3
% Notes:- plot description of example jet_m. 
% 	- #_plot > #_ntuple? holds for leptonic entities: difference in number represents reconstructed leptons
% 	- trigger and reconstruction requirements are maybe found in Jana_Novakova book ore elsewhere in the literature
% 	- pT>17GeV (otherwise too much background) and eta<2,47 (interested in ``central'' letpons), phi has no constraints due to rotational invariance of the detector
% 	- 3.1a not sure what they mean, but maybe: all-hadronic: bbqq; lepton+jets: bbql; dilepton: bbll. b=b-quark, q=any lighter quark, l=lepton.
% 	- 3.1b if-environments in runSelection.C
% 	- 3.1c ttbar, diboson, ww, zz, singletop - topology: bbql for amount of events and detectability. 
% 	- 3.1e maybe include the table in output.info.txt
% 	- 3.1g efficiencies from outputRunselection/; requirements: jetGood>lepN>met>btagged>jetN (with rising efficiency) for more or less all background processes
% 	- efficiency for Z'(1000) is 21% 
% 	- 3.1h main selected background process is ttbar with 10e5 events compared to the others being of order 10e3 or less
	

\section{Fundamental and derived quantities}
4+5
% Notes:- 4a+b: comparing a data set with ttbar
% 	- taking inv_massSys as most powerful discriminant because it has resonances (peaks) at the particles or systems mass which is known (ttbar) or seeked (Z')

\section{Data-MC comparisons and constraints on $Z'$-mass}
6+7
% Notes:- 6.2c: some have fluctuations -> nothing to worry about. But in jet_eta, jet_pt and lep_eta for example the data points are almost always below the MC-expectation by fairly the same amount ->maybe there is a small, overall factor missing
% 	- 6.2d: from inv_massSys no Z' signal can be seen by eye.
% 	- 7.1a: from results/notes we get a p-value of 0.46. For 95 C.F. you get chi=36.4149 which means that the impact of the Z' cross section gets increased until this chi is reached.
% 	- 7.1d: by doing this, the lighter Z' overproduce the observed jets up to the mass of 850 GeV above which the Z' boson is still viable

\section{Conclusion}

 \begin{thebibliography}{WissOnl}
 	\bibitem{Anl} TU Dortmund instruction for experiment Nr.60 \url{http://129.217.224.2/HOMEPAGE/Anleitung_FPBSc.html}
 	\bibitem{ref} Laser spectroscopy of Rubidium \\ \url{http://www.photodigm.com/literature/applications-notes/rubidium-absorption-spectroscopy}
 \end{thebibliography}

% ========================================
%	Literaturverzeichnis
% ========================================

%\bibliographystyle{plainnat}			% Bibliographie-Style auswählen
%\bibliography{BIBDATEI}			% Literaturverzeichnis

% ========================================
%	Das Dokument endent
% ========================================
\end{document}
