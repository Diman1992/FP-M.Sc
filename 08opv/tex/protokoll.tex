% ========================================
%	Header einbinden
% ========================================

\input{longheader.tex}
\usepackage{xfrac}
\usepackage{xcolor}
\usepackage{setspace}\usepackage{threeparttable}
\usepackage{fancyhdr}
\usepackage{graphicx}
\usepackage[official]{eurosym}
\usepackage{geometry}
\newcommand{\ti}{\text{i}}
\usepackage{siunitx}
\sisetup{output-decimal-marker = {,}}

% ========================================
%	Angaben für das Titelblatt
% ========================================

\title{V51 - Schaltungen mit Operationsverstärkern\\
\hspace{15cm}% Titel des Versuchs 
	\large TU Dortmund, Fakultät Physik\\ 
	\normalsize Fortgeschrittenen-Praktikum}

\author{Jan Latarius\\			% Name Praktikumspartner A
	{\small \href{jan.latarius@tu-dortmund.de}{jan.latarius@tu-dortmund.de}}	% Erzeugt interaktiven einen Link
	\and						% um einen weiteren Author hinzuzfügen
	Dimitrios Skodras\\					% Name Praktikumspartner B
	{\small \href{dimitrios.skodras@tu-dortmund.de}{dimitrios.skodras@tu-dortmund.de}}		% Erzeugt interaktiven einen Link
}
\date{09.01.2017}				% Das Datum der Versuchsdurchführung

% ========================================
%	Das Dokument beginnt
% ========================================

\begin{document}
	
% ========================================
%	Titelblatt erzeugen
% ========================================

\maketitle					% Jetzt wird die Titelseite erzeugt
\thispagestyle{empty} 				% Weder Kopfzeile noch Fußzeile

% ========================================
%	Der Vorspann
% ========================================

%\newpage					% Wenn Verzeichnisse auf einer neuen Seite beginnen sollen
%\pagestyle{empty}				% Weder Kopf- noch Fußzeile für Verzeichnisse

\tableofcontents

%\newpage					% eine neue Seite
%\thispagestyle{empty}				% Weder Kopf- noch Fußzeile für Verzeichnisse
%\listoffigures					% Abbildungsverzeichnis

%\newpage					% eine neue Seite
%\thispagestyle{empty}				% Weder Kopf- noch Fußzeile für Verzeichnisse
%\listoftables					% Tabellenverzeichnis
\newpage					% eine neue Seite


% ========================================
%	Kapitel
% ========================================

\section{Theoretische Grundlagen}
\section{Durchführung}

\section{Auswertung}
\section{Diskussion}


\newpage
 \begin{thebibliography}{WissOnl}
 	\bibitem{Anl} TU Dortmund Anleitung für Versuch Nr.51 \url{http://129.217.224.2/HOMEPAGE/Anleitung_FPBSc.html}
 	\end{thebibliography}

% ========================================
%	Literaturverzeichnis
% ========================================

%\bibliographystyle{plainnat}			% Bibliographie-Style auswählen
%\bibliography{BIBDATEI}			% Literaturverzeichnis

% ========================================
%	Das Dokument endent
% ========================================
\end{document}
