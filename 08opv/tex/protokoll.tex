% ========================================
%	Header einbinden
% ========================================

\input{longheader.tex}
\usepackage{xfrac}
\usepackage{xcolor}
\usepackage{setspace}\usepackage{threeparttable}
\usepackage{fancyhdr}
\usepackage{graphicx}
\usepackage[official]{eurosym}
\usepackage{geometry}
\newcommand{\ti}{\text{i}}
\usepackage{siunitx}
\sisetup{output-decimal-marker = {,}}

% ========================================
%	Angaben für das Titelblatt
% ========================================

\title{V51 - Schaltungen mit Operationsverstärkern\\
\hspace{15cm}% Titel des Versuchs 
	\large TU Dortmund, Fakultät Physik\\ 
	\normalsize Fortgeschrittenen-Praktikum}

\author{Jan Latarius\\			% Name Praktikumspartner A
	{\small \href{jan.latarius@tu-dortmund.de}{jan.latarius@tu-dortmund.de}}	% Erzeugt interaktiven einen Link
	\and						% um einen weiteren Author hinzuzfügen
	Dimitrios Skodras\\					% Name Praktikumspartner B
	{\small \href{dimitrios.skodras@tu-dortmund.de}{dimitrios.skodras@tu-dortmund.de}}		% Erzeugt interaktiven einen Link
}
\date{09.01.2017}				% Das Datum der Versuchsdurchführung

% ========================================
%	Das Dokument beginnt
% ========================================

\begin{document}
	
% ========================================
%	Titelblatt erzeugen
% ========================================

\maketitle					% Jetzt wird die Titelseite erzeugt
\thispagestyle{empty} 				% Weder Kopfzeile noch Fußzeile

% ========================================
%	Der Vorspann
% ========================================

%\newpage					% Wenn Verzeichnisse auf einer neuen Seite beginnen sollen
%\pagestyle{empty}				% Weder Kopf- noch Fußzeile für Verzeichnisse

\tableofcontents

%\newpage					% eine neue Seite
%\thispagestyle{empty}				% Weder Kopf- noch Fußzeile für Verzeichnisse
%\listoffigures					% Abbildungsverzeichnis

%\newpage					% eine neue Seite
%\thispagestyle{empty}				% Weder Kopf- noch Fußzeile für Verzeichnisse
%\listoftables					% Tabellenverzeichnis
\newpage					% eine neue Seite


% ========================================
%	Kapitel
% ========================================

\section{Theoretische Grundlagen}
Der Operationsverstärker (OPV) ist ein Halbleiterbauelement mit integrierter Schaltung. Er dient als
Differenzverstärker sowohl für Gleichspannung als auch für Wechselspannung. Der OPV ist so konstruiert, 
dass sein Verhalten von der äußeren Beschaltung beeinflusst werden kann. In diesem Versuch soll der OPV 
in seinen verschiedenen äußeren Beschaltungen kennengelernt und untersucht werden.
\subsection{Eigenschaften des OPVs}
Die Ausgangsspannung $U_A$ des OPVs ist proportional zur Spannungsdifferenz an den beiden Eingängen, wobei
die Ausgangsspannung begrenzt wird durch die Betriebsspannung,
\begin{align}
 U_A = V(U_P - U_N), \quad -U_B < U_A < U_B.
 \label{eq:Uaus}
\end{align}
Der Faktor $V$ bezeichnet hierbei die Verstärkung des OPVs. Die Schaltung des OPV ist in Abbildung \ref{pic:basic}
dargestellt. 
\begin{figure}[t]
 \includegraphics[width = 0.5\textwidth]{../pics/OPVBasic.png}
 \caption{Anschlüsse des Operationsverstärkers \cite{Anl}.}
 \label{pic:basic}
\end{figure}
Der mit $``+''$ gekennzeichnete Eingang wird nicht-invertierender Eingang genannt, der mit $``-''$ hingegen
invertierend. Ist das Potenzial an dem nicht-invertierenden Eingang größer als am Invertierenden,
so ist $U_A$ in Phase mit der Spannungsdifferenz $U_P-U_N$, im umgekehrten Fall gegenphasig. Die Kennlinie
des OPV ist in Abbildung \ref{pic:linie} mit übertrieben niedriger Steigung dargestellt.
\begin{figure}[t]
 \includegraphics[width = 0.5\textwidth]{../pics/OPVLinie.png}
 \caption{Kennlinie des Operationsverstärkers \cite{Anl}.}
 \label{pic:linie}
\end{figure}
Zur Berechnung von Schaltungen mit dem OPV betrachtet man zur Vereinfachung einen idealen OPV. 
Dieser ist dadurch gekennzeichnet, dass seine Leerlaufverstärkung sowie die Eingangswiderstände unendlich hoch
sind und der Ausgangswiderstand gleich null ist. Diese Größen sind beim realen OPV natürlich
nicht mehr infitesimal. Will man Schaltungen mit OPVs genauer beschreiben,
so müssen weitere Größen betrachtet werden. Diese lauten
\begin{itemize}
 \item Gleichtaktverstärkung: Wenn $U_P = U_N$, dann ist $U_A$ dennoch ungleich null aufgrund von Asymmetrien 
 der integrierten Schaltung des OPV.
 \item Eingangsruhestrom: Dem endlichen Eingangswiderstand folgt auch ein geringer Eingangsstrom.
 \item Offsetstrom: Dies ist die Differenz der beiden Eingangsströme.
 \item Differenzeingangs- und Gleichtaktwiderstand: Mit den Eingangsruheströmen wird der Differenzeingangswiderstand
 gemessen, während eine Eingangsspannung null ist. Der Gleichtaktwiderstand ergibt sich bei gleichen Eingangsspannungen
 und der Summe der Eingangsruheströme.
 \item Offsetspannung: Hiermit wird die Asymmetrie der Eingangsspannungen beschrieben. Die Offsetspannung ist diejenige
 Spannung, welche eingestellt werden muss, damit die Ausgangsspannung gleich null ist.
 \end{itemize}
 \subsection{Linearverstärker}
 Die Sättigung beim OPV erfolgt aufgrund seiner sehr hohen Verstärkung bereits bei geringen Ansteuerungen. 
 Um die Ausgangsspannung 
 zu begrenzen und zu kontrollieren,
wird ein Teil der Ausgangsspannung an den invertierenden Eingang zurückgegeben, sodass
die Eingangsspannung verringert und somit
auch die Ausgangsspannung begrenzt wird.
Dies wird mit Gegenkopplung bezeichnet. Die
Schaltung der Linearverstärkers ist in Abbildung \ref{pic:linear}
dargestellt. 
\begin{figure}[t]
 \includegraphics[width = 0.5\textwidth]{../pics/Linear.png}
 \caption{Schaltung des Linearverstärkers \cite{Anl}.}
 \label{pic:linear}
\end{figure}
Die Verstärkung lässt sich mithilfe
des idealen OPVs wie folgt berechnen. Die Eingangsspannung des idealen OPVs ist
unendlich. Daher ist der Strom $I_N$ gleich null. An dem Punkt A in Abbildung \ref{pic:linear}
addieren sich die Ströme gerade zu null. Somit gilt
\begin{align}
 0 = \frac{U_1}{R_1} + \frac{U_A}{R_N}.
\end{align}
Und die Verstärkung ergibt sich damit zu
\begin{align}
 V' = \frac{U_A}{U_1} = -\frac{R_N}{R_1}.
 \label{eq:Vprime}
\end{align}
Bei dem realen OPV bewirken die oben genannten Größen geringfügige Korrekturen.
In diesem Versuch soll nur der Einfluss der endlichen Leerlaufverstärkung betrachtet werden.
Zur Herleitung der Leerlaufverstärkung $V'$ wird der Spannungsteiler $R_1,R_N$ betrachtet.
Für diesen gilt
\begin{align}
 \frac{U_N-U_1}{U_A-U_1} = \frac{R_1}{R_1+R_N}.
\end{align}
Mit \eqref{eq:Uaus} ergibt sich die Leerlaufverstärkung zu
\begin{align}
 \frac{1}{V'} =& -\frac{U_1}{U_A} = \frac{1}{V} + \frac{R_1}{R_N}\left(1+\frac{1}{V}\right) \\
 \approx& \frac{1}{V} + \frac{R_1}{R_N}.
\end{align}
Für $V\gg R_N/R_1$ ergibt sich \eqref{eq:Vprime}. 
Damit zeigt sich, dass die Verstärkung 
auch für einen realen OPV nur von
dem Widerstandsverhältnis abhängt, wenn eine 
große Gegenkopplung gewählt wird. Zudem 
wird die Stabilität des Verstärkers erhöht,
da der Einfluss der Temperatur und die damit 
verbundenen Schwankungen von V nahezu
verschwinden. Die Gegenkopplung wirkt sich
zudem auf weitere Parameter des realen OPV
aus. So wird der Ausgangswiderstand verkleinert 
und die Bandbreite erhöht.

\subsection{Umkehr-Integrator}
Die Schaltung des Umkehr-Integrators ist in
Abbildung \ref{pic:umInt} dargestellt. 
\begin{figure}[t]
 \includegraphics[width = 0.5\textwidth]{../pics/umkehrInt.png}
 \caption{Schaltung des Umkehr-Integrators \cite{Anl}.}
 \label{pic:umInt}
\end{figure}
Anstelle des Rückkopplungswiderstands ist hier ein Kondensator ein-
gebaut. Die Funktionsweise als Integrator ist
wie folgt zu sehen.
Der Strom, der über den Widerstand in den
Punkt A fließt, fließt nun auch in den Kon-
densator. Zudem wird die Ausgangsspannung
entegegen der Stromrichtung gemessen. Somit
gilt
\begin{align}
 I_C = \frac{U_1}{R} = -C \frac{\dx U_A}{\dx t}.
\end{align}
Dividieren durch $C$ und integrieren führt zu
\begin{align}
 U_A = -\frac{1}{RC} \in U_1 \dx t\,.
\end{align}

\subsection{Umkehr-Differenziator}
Die Schaltung des Differentiators ist in Abbildung \ref{pic:umDiff}
dargestellt.
\begin{figure}[t]
 \includegraphics[width = 0.5\textwidth]{../pics/umkehrDiff.png}
 \caption{Schaltung des Umkehr-Integrators \cite{Anl}.}
 \label{pic:umDiff}
\end{figure}
Hier sind Widerstand 
und Kondensator im Vergleich zum
Integrator vertrauscht. Die Herleitung der Wirkungsweise 
als Differentiator ist analog zu der
des Integrators. Es ergibt sich die Ausgangsspannung zu
\begin{align}
 U_A = -RC \frac{\dx U_1}{\dx t}.
\end{align}

\subsection{Schmitt-Trigger}
Der Schmitt-Trigger besitzt im Gegensatz zu
den vorherigen Schaltungen eine Mitkopplung. 
Dabei wird die Ausgangsspannung über
einen Widerstand an den nicht-invertierenden
Eingang zurückgegeben. Die Schaltung ist in
Abbildung \ref{pic:schmitt} gezeigt. 
\begin{figure}[t]
 \includegraphics[width = 0.5\textwidth]{../pics/schmitt.png}
 \caption{Schaltung des Umkehr-Integrators \cite{Anl}.}
 \label{pic:schmitt}
\end{figure}
Hier wird die Eingangsspannung 
verstärkt und gleichphasig zurück auf
den Eingang gegeben. Somit wird die erhöhte
Spannung noch weiter verstärkt. Dieses Verhalten 
wird ausgenutzt, um den Schmitt-Trigger
als Schalter zu verwenden. Ist hier eine bestimmte 
Schwellspannung erreicht, so springt
die Ausgangsspannung auf die positive oder
entsprechend negative Betriebspannung. Die
Schelle ist dabei durch
\begin{align}
 \frac{R_1}{R_P} U_B
\end{align}
gegeben und der doppelte Wert wird Schalthysterese bezeichnet.

\subsection{Signalgenerator}
Der Operationsverstärker kann ebenfalls verwendet werden, um verschiedene Formen
von Wechselspannungen zu generieren. Möglichkeiten zur Kombination der OPVs 
zu diesem Zweck sind in den Abbildungen \ref{pic:dreiRect}
und \ref{pic:sinus} dargestellt.
Für einen Rechteckgenerators wird die oszillierende Rechteckspannung durch einen
Schmitt-Trigger erzeugt. Mithilfe eines Integrators kann dieses Signal 
zu einer Dreiecksspannung umgewandelt werden.

\begin{figure}[t]
 \includegraphics[width = 0.9\textwidth]{../pics/DreiRect.png}
 \caption{Schaltung des Umkehr-Integrators \cite{Anl}.}
 \label{pic:dreiRect}
\end{figure}

\begin{figure}[t]
 \includegraphics[width = 0.9\textwidth]{../pics/sinus.png}
 \caption{Schaltung des Umkehr-Integrators \cite{Anl}.}
 \label{pic:sinus}
\end{figure}

\subsection{Logarithmierer und Exponentialgenerator}
\begin{figure}[t]
 \includegraphics[width = 0.5\textwidth]{../pics/log.png}
 \caption{Schaltung des Umkehr-Integrators \cite{Anl}.}
 \label{pic:log}
\end{figure}

\begin{figure}[t]
 \includegraphics[width = 0.5\textwidth]{../pics/exp.png}
 \caption{Schaltung des Umkehr-Integrators \cite{Anl}.}
 \label{pic:exp}
\end{figure}


\section{Durchführung}

\section{Auswertung}
\section{Diskussion}


\newpage
 \begin{thebibliography}{WissOnl}
 	\bibitem{Anl} TU Dortmund Anleitung für Versuch Nr.51 \url{http://129.217.224.2/HOMEPAGE/Anleitung_FPBSc.html}
 	\end{thebibliography}

% ========================================
%	Literaturverzeichnis
% ========================================

%\bibliographystyle{plainnat}			% Bibliographie-Style auswählen
%\bibliography{BIBDATEI}			% Literaturverzeichnis

% ========================================
%	Das Dokument endent
% ========================================
\end{document}
