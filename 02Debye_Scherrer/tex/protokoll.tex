% ========================================
%	Header einbinden
% ========================================

\input{longheader.tex}

% ========================================
%	Angaben für das Titelblatt
% ========================================

\title{Debye-Scherrer Aufnahmen\\				% Titel des Versuchs 
\large TU Dortmund, Fakultät Physik\\ 
\normalsize Fortgeschrittenen-Praktikum}

\author{Jan Adam\\			% Name Praktikumspartner A
{\small \href{jan.adam@tu-dortmund.de}{jan.adam@tu-dortmund.de}}	% Erzeugt interaktiven einen Link
\and						% um einen weiteren Author hinzuzfügen
Dimitrios Skodras\\					% Name Praktikumspartner B
{\small \href{dimitrios.skodras@tu-dortmund.de}{dimitrios.skodras@tu-dortmund.de}}		% Erzeugt interaktiven einen Link
}
\date{27. Mai 2015}				% Das Datum der Versuchsdurchführung

% ========================================
%	Das Dokument beginnt
% ========================================

\begin{document}

% ========================================
%	Titelblatt erzeugen
% ========================================

\maketitle					% Jetzt wird die Titelseite erzeugt
\thispagestyle{empty} 				% Weder Kopfzeile noch Fußzeile

% ========================================
%	Der Vorspann
% ========================================

%\newpage					% Wenn Verzeichnisse auf einer neuen Seite beginnen sollen
%\pagestyle{empty}				% Weder Kopf- noch Fußzeile für Verzeichnisse

\tableofcontents

%\newpage					% eine neue Seite
%\thispagestyle{empty}				% Weder Kopf- noch Fußzeile für Verzeichnisse
%\listoffigures					% Abbildungsverzeichnis

%\newpage					% eine neue Seite
%\thispagestyle{empty}				% Weder Kopf- noch Fußzeile für Verzeichnisse
%\listoftables					% Tabellenverzeichnis
\newpage					% eine neue Seite


% ========================================
%	Kapitel
% ========================================

%\section{Einleitung}				% Bei Bedarf

\section{Theorie}\label{sec:theorie}
Viele makroskopische Eigenschaften von Materie können auf den Aufbau ihrer Grundbausteine, dh. Atome und Moleküle zurückgeführt werden. Bei Feststoffen liegen diese in kristalliner Form vor und bilden regelmäßig auftretende Strukturen, sog. Gitter aus. Durch die Untersuchung dieser Gitter können wichtige tensorielle und vektorielle makroskopische Eigenschaften, wie zum Beispiel Elastizität und Permeabilität bestimmt werden. Um die Gitterabstände auflösen zu können, die im $\mathring{A}$ngströmbereich liegen, ist die Wellenlänge des sichtbaren Lichtes zu groß. Röntgenstrahlen haben jedoch das benötigte Auflösungsvermögen.

\subsection{Über die Struktur der Materie}
Wie in \ref{sec:theorie} erwähnt, liegen die Atome von Festkörpern in Gitterstruktur vor. Diese können durch einen Translationsvektor
\begin{align}
	\vec{t} = n_1\vec{a} +n_2\vec{b} + n_3\vec{c}
\end{align}
beschrieben werden, der ausgehen von einem Startpunkt auf alle anderen Gitterpunkte zeigt, wenn man für die $n_i$ ganze Zahlen einsetzt. An den Gitterpunkten können entweder einzelne oder mehrere Atome sitzen. In letzterem Fall nennt man diese Ansammlung Basis. 

Da es für die $n_i$ keine Begrenzung gibt, existieren unendlich viele verschiedene Gitter. Hinsichtlich ihrer Symmetrieeigenschaften können sie jedoch im dreidimensionalen in 14 Gitterstrukturen zusammengefasst werden, den sog. Bravais-Gittern. Diese sind in 7 Gittersysteme unterteilt: triklin, monoklin, (ortho-)rhombisch, tetragonal, rhomboedrisch, hexagonal und kubisch. Alle Systeme und ihre Eigenschaften sind in Abbildung \ref{pic:bravais} aufgelistet.
\begin{figure}[htbp]
	\includegraphics[width=0.9\textwidth]{../pics/bravais.png}
	\caption{Auflistung der 14 Bravaisgitter im dreidimensionalen Raum.}
	\label{pic:bravais}
\end{figure}

Besonderes Interesse  kommt dabei der sog. Elementarzelle zu. Dies ist die kleinste Einheit, die eine Kristallstruktur vollkommen festlegt. Beispielsweise kann man dazu das von den Vektoren a , b und c aufgespannte Parallelepiped benutzen. Liegen nur auf den Eckpunkten Atome, enthält die Zelle also im Ganzen nur ein einzelnes Atom, so nennt man sie primitiven Elementarzelle. 
Nicht jede Kristallstruktur lässt sich jedoch durch Vervielfachung einer primitiven Elementarzelle aufbauen. Die in diesem Versuch untersuchten Stoffe haben eine kubische Grundstruktur, daher wird nur dieses System im Folgenden detaillierter beschrieben. In Abbildung \ref{pic:gitterTyp}

\begin{figure}[htbo]
	\centering
	\begin{subfigure}[b]{0.3\textwidth}
		\includegraphics[width=\textwidth]{../pics/bcc.png}
		\caption{bcc-Gitter}
		\label{fig:bcc}
	\end{subfigure}
	~ %add desired spacing between images, e. g. ~, \quad, \qquad, \hfill etc.
	%(or a blank line to force the subfigure onto a new line)
	\begin{subfigure}[b]{0.3\textwidth}
		\includegraphics[width=\textwidth]{../pics/fcc.png}
		\caption{fcc-Gitter}
		\label{pic:fcc}
	\end{subfigure}
	\caption{Verschiedene}
	\label{pic:gitterTyp}
\end{figure}




\section{Theorie}

\section{Durchführung}

\section{Auswertung}

\parskip 340pt
\Large{Literatur}\\\\

% ========================================
%	Literaturverzeichnis
% ========================================

%\bibliographystyle{plainnat}			% Bibliographie-Style auswählen
%\bibliography{BIBDATEI}			% Literaturverzeichnis

% ========================================
%	Das Dokument endent
% ========================================

\end{document}
