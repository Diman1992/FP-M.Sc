% ========================================
%	Header einbinden
% ========================================

\input{longheader.tex}

% ========================================
%	Angaben für das Titelblatt
% ========================================

\title{Debye-Scherrer Aufnahmen\\				% Titel des Versuchs 
\large TU Dortmund, Fakultät Physik\\ 
\normalsize Fortgeschrittenen-Praktikum}

\author{Jan Adam\\			% Name Praktikumspartner A
{\small \href{jan.adam@tu-dortmund.de}{jan.adam@tu-dortmund.de}}	% Erzeugt interaktiven einen Link
\and						% um einen weiteren Author hinzuzfügen
Dimitrios Skodras\\					% Name Praktikumspartner B
{\small \href{dimitrios.skodras@tu-dortmund.de}{dimitrios.skodras@tu-dortmund.de}}		% Erzeugt interaktiven einen Link
}
\date{27. Mai 2015}				% Das Datum der Versuchsdurchführung

% ========================================
%	Das Dokument beginnt
% ========================================

\begin{document}

% ========================================
%	Titelblatt erzeugen
% ========================================

\maketitle					% Jetzt wird die Titelseite erzeugt
\thispagestyle{empty} 				% Weder Kopfzeile noch Fußzeile

% ========================================
%	Der Vorspann
% ========================================

%\newpage					% Wenn Verzeichnisse auf einer neuen Seite beginnen sollen
%\pagestyle{empty}				% Weder Kopf- noch Fußzeile für Verzeichnisse

\tableofcontents

%\newpage					% eine neue Seite
%\thispagestyle{empty}				% Weder Kopf- noch Fußzeile für Verzeichnisse
%\listoffigures					% Abbildungsverzeichnis

%\newpage					% eine neue Seite
%\thispagestyle{empty}				% Weder Kopf- noch Fußzeile für Verzeichnisse
%\listoftables					% Tabellenverzeichnis
\newpage					% eine neue Seite


% ========================================
%	Kapitel
% ========================================

%\section{Einleitung}				% Bei Bedarf

\section{Theorie}\label{sec:theorie}
Viele Eigenschaften von Materie können auf den Aufbau ihrer Grundbausteine, dh. Atome und Moleküle zurückgeführt werden. Bei Feststoffen liegen diese in kristalliner Form vor und bilden regelmäßig auftretende Strukturen, sog. Gitter aus. Durch die Untersuchung dieser Gitter können wichtige tensorielle und vektorielle makroskopische Eigenschaften, wie zum Beispiel Elastizität und Permeabilität bestimmt werden. Um die Gitterabstände auflösen zu können, die im Angströmbereich liegen, ist die Wellenlänge des sichtbaren Lichtes zu groß. Röntgenstrahlen haben jedoch das benötigte Auflösungsvermögen.

\subsection{Über die Struktur der Materie}
Wie in \ref{sec:theorie} erwähnt, liegen die Atome von Festkörpern in Gitterstruktur vor. Diese können durch einen Translationsvektor
\begin{align}
	\vec{t} = n_1\vec{a} +n_2\vec{b} + n_3\vec{c}
\end{align}
beschrieben werden, der ausgehen von einem Startpunkt auf alle anderen Gitterpunkte zeigt, wenn man für die $n_i$ ganze Zahlen einsetzt. An den Gitterpunkten können entweder einzelne oder mehrere Atome sitzen. In letzterem Fall nennt man diese Ansammlung Basis. 




\section{Theorie}

\section{Durchführung}

\section{Auswertung}

\parskip 340pt
\Large{Literatur}\\\\

% ========================================
%	Literaturverzeichnis
% ========================================

%\bibliographystyle{plainnat}			% Bibliographie-Style auswählen
%\bibliography{BIBDATEI}			% Literaturverzeichnis

% ========================================
%	Das Dokument endent
% ========================================

\end{document}
