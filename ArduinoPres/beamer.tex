
\documentclass[hyperref={pdfpagelabels=false}]{beamer}
% Die Hyperref Option hyperref={pdfpagelabels=false} verhindert die Warnung:
% Package hyperref Warning: Option `pdfpagelabels' is turned off
% (hyperref)                because \thepage is undefined. 
% Hyperref stopped early 
%
% \input{longheader.tex}
% \input{global.tex}

\usepackage{lmodern}
% Das Paket lmodern erspart die folgenden Warnungen:
% LaTeX Font Warning: Font shape `OT1/cmss/m/n' in size <4> not available
% (Font)              size <5> substituted on input line 22.
% LaTeX Font Warning: Size substitutions with differences
% (Font)              up to 1.0pt have occurred.
%

% % % % % % % % % % % % % % % % % % % % % % % % % % % % % % % % % % % % % % % % % % % %
\usepackage{siunitx}
\sisetup{load-configurations=abbreviations}
\sisetup{
	%locale=DE,
	seperr=true,                    % Fehler anzeigen
	tightpm,                        % Abstand zwischen Fehler verringern
	tophrase={{\text{ bis }}},
	fraction=nice,
	per-mode=fraction,
	free-standing-units=true,
	space-before-unit=true,
	use-xspace=true,
	group-separator={{\text{~}}},
	list-final-separator={{\text{ und }}}
}
\usepackage{natbib}
\usepackage[labelformat=empty]{caption}
\usepackage{movie15}
\usepackage{xcolor,colortbl}
\usepackage{slashed}
\usepackage{amsfonts}
\usepackage{amssymb}
\usepackage{amsmath}
\usepackage{amscd}
\usepackage{amstext}
\usepackage[ngerman,german]{babel, varioref}
\usepackage[T1]{fontenc}
\usepackage[utf8]{inputenc}
\usepackage{xfrac}
\usepackage{booktabs}

% % % % % % % % % % % % % % % % % % % % % % % % % % % % % % % % % % % % % % % % % % % % % % % % %
% Wenn \titel{\ldots} \author{\ldots} erst nach \begin{document} kommen,
% kommt folgende Warnung:
% Package hyperref Warning: Option `pdfauthor' has already been used,
% (hyperref) ... 
% Daher steht es hier vor \begin{document}

\title[Arduino Vortrag]{Arduino: Helligkeitsempfindliche Radleuchte}  
\institute{Fortgeschrittenen Praktikum\\
Technische Universit\"at Dortmund}
\author{Dimitrios Skodras} 
\date{21.11.2016} 

% zusaetzlich ist das usepackage{beamerthemeshadow} eingebunden 
\usepackage{beamerthemeshadow}


%  \beamersetuncovermixins{\opaqueness<1>{25}}{\opaqueness<2->{15}}
%  sorgt dafuer das die Elemente die erst noch (zukuenftig) kommen 
%  nur schwach angedeutet erscheinen 
\beamersetuncovermixins{\opaqueness<1>{25}}{\opaqueness<2->{15}}
% klappt auch bei Tabellen, wenn teTeX verwendet wird\ldots

\beamertemplatenavigationsymbolsempty

\begin{document}

\setbeamertemplate{footline}
{%
  \leavevmode%
 \begin{beamercolorbox}%
    [wd=.5\paperwidth,ht=2.5ex,dp=1.125ex,leftskip=.3cm,rightskip=.3cm]%
    {author in head/foot}%
    \usebeamerfont{author in head/foot}%
    \hfill\insertshortauthor
  \end{beamercolorbox}%
  \begin{beamercolorbox}%
    [wd=.5\paperwidth,ht=2.5ex,dp=1.125ex,leftskip=.3cm ,rightskip=.3cm]%
    {title in head/foot}%
    \usebeamerfont{title in head/foot}%
    \insertshorttitle\hfill\insertframenumber{}
  \end{beamercolorbox}%
}%

\setbeamertemplate{caption}{\raggedright\insertcaption\par}
\captionsetup[figure]{font=small,skip=0pt}
\begin{frame}
\titlepage
\end{frame} 

% \begin{frame}
% \frametitle{Gliederung}
% \tableofcontents
% \end{frame} 
% 
% \newcommand{\tmotiv}{Konzept}
% \section{\tmotiv}
% 

 \begin{frame}
  \frametitle{Motivation}
  Wie kann ich möglichst alle zur Verfügung stehende Bauteile miteinbinden?
  \vspace{0.3cm}
  
  
  \begin{minipage}{0.49\textwidth}
   Input:
   \begin{itemize}
    \item Photodiode
    \item Button
   \end{itemize}
  \end{minipage}
    \begin{minipage}{0.49\textwidth}
    Output:
   \begin{itemize}
    \item 2 LED
    \item 1 RGB LED
   \end{itemize}
  \end{minipage}

 \end{frame}
 
 \begin{frame}
  \frametitle{Grobe Idee}
  \begin{itemize}
   \item Helligkeitsempfindliche Fahrradlampe
   \item verschiedene Blinkfrequenzen durch Knopfdruck
   \item mögliche, weitere Leuchtbedingung: Bewegung des Rads
   \item[$\rightarrow$] Lampe muss bei eintretender Dunkelheit (ggf. während der Fahrt) nicht angeschaltet werden.
  \end{itemize}
 \end{frame}


 \begin{frame}
 \frametitle{Konzept}
 \begin{itemize}
  \item Photodiode misst Helligkeit
  \item Button erhöht die Blinkfrequenz von nicht leuchtend zu dauerhaft leuchtend in 5 Stufen
  \item Weiße LED bestätigt Knopfdruck
  \item Gelbe LED zeigt in Helligkeit derzeitigen Status an
  \item RGB LED leuchtet bei Dunkelheit und Frequenzstatus$>0$ in zufälligen Farbkombinationen
 \end{itemize}

 \end{frame}

 
 \begin{frame}
  \frametitle{Schaltplan}
  \includegraphics[scale=0.15]{Schaltung.jpg}
 \end{frame}




\end{document}
