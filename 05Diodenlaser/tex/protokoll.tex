% ========================================
%	Header einbinden
% ========================================

\input{longheader.tex}

% ========================================
%	Angaben für das Titelblatt
% ========================================

\title{V60 - Der Diodenlaser\\				% Titel des Versuchs 
	\large TU Dortmund, Fakultät Physik\\ 
	\normalsize Fortgeschrittenen-Praktikum}

\author{Jan Adam\\			% Name Praktikumspartner A
	{\small \href{jan.adam@tu-dortmund.de}{jan.adam@tu-dortmund.de}}	% Erzeugt interaktiven einen Link
	\and						% um einen weiteren Author hinzuzfügen
	Dimitrios Skodras\\					% Name Praktikumspartner B
	{\small \href{dimitrios.skodras@tu-dortmund.de}{dimitrios.skodras@tu-dortmund.de}}		% Erzeugt interaktiven einen Link
}
\date{06.06.2016}				% Das Datum der Versuchsdurchführung

% ========================================
%	Das Dokument beginnt
% ========================================

\begin{document}
	
% ========================================
%	Titelblatt erzeugen
% ========================================

\maketitle					% Jetzt wird die Titelseite erzeugt
\thispagestyle{empty} 				% Weder Kopfzeile noch Fußzeile

% ========================================
%	Der Vorspann
% ========================================

%\newpage					% Wenn Verzeichnisse auf einer neuen Seite beginnen sollen
%\pagestyle{empty}				% Weder Kopf- noch Fußzeile für Verzeichnisse

\tableofcontents

%\newpage					% eine neue Seite
%\thispagestyle{empty}				% Weder Kopf- noch Fußzeile für Verzeichnisse
%\listoffigures					% Abbildungsverzeichnis

%\newpage					% eine neue Seite
%\thispagestyle{empty}				% Weder Kopf- noch Fußzeile für Verzeichnisse
%\listoftables					% Tabellenverzeichnis
\newpage					% eine neue Seite


% ========================================
%	Kapitel
% ========================================

\section{Theoretical Essentials}
This experiment aims to set up and fine tune a diode laser to match the absorbtion wavelength of rubidium - \SI{780}{nm}. A diode laser is a semiconductor element that emits coherent light but in contrast to a regular solid-state laser its wavelength can be changed almost freely.
\subsection{Composition}
The laser basically consists of these components:
\begin{itemize}
	\item Laser diode
	\begin{itemize}
		\item Active layer
		\item Cavity mirrors
	\end{itemize}
	\item Diffraction grating
\end{itemize}

The laser diode operates with direct current and emits light in the so called active layer where electrons and holes of the semiconductor recombine into photons. The walls of the cavity reflects most of the light which results in standing waves inside this resonator. The direct current creates more and more electron-hole pairs resulting in occupation inversion and finally in stimulated emission of coherent photons.

The wavelength of these photons mainly depend on the bandgap of the semiconductor. The width of the spectrum is very wide compared to regular lasers and also the laser diode is very sensitive to feedback of emitted light which can easily result in instabilities. 

As counteract only a single wavelength of the emitted light gets reflected into the diode by placing a collimator lens and a diffraction grating into the light beam. The grating splits up the beam into its different wavelengths and reflects only a single one directly into the diode. Since the wavelength is much smaller than the width of the cavity ($\lambda \ll b$) almost every $\lambda$ can create standing waves. By reflecting only 15\% of the emitted light back into the laser the emission stabilises and the spectral width narrows down to less than \SI{1}{MHz} which is smaller than the spectral width of atomic transitions $(\sim \SI{5}{MHz})$.

\section{Laser configuration}
This section describes how to manipulate the wavelength of the laser diode to emit light that resonates with rubidium $(\sim \SI{780}{nm})$. A laser will basically always operate in the mode that has the highest net gain since more light of this mode will be reflected back into the cavity resulting in more stimulated emission of equal photons that in return get fed back into the diode resulting in a rapid increase of photons with the corresponding wavelength. There are several factors that have an impact on the gain function and these are displayed in \mbox{figure \ref{fig:gains}}.

\subsection{Medium gain}
As mentioned before the photon energy and therefore the wavelength depends on the bandgap which in return depends on the temperature. This peak however is very wide and it is therefore unnecessary to fine tune the temperature to match the rubidium exactly. The heating is done indirectly by changing the direct current.

\subsection{Grating feedback}
The gain of the grating has a maximum that is narrower and matches the reflected wavelength. There is only a single peak because only the 1. order reflection is scattered back into the diode.

\subsection{Internal and external cavity}
The internal cavity is the diode itself and its gain is periodic in frequency. The period is called 'free spectral range' and its size is:
\begin{align}
	\nu_\text{FSR} = \frac{c}{2Ln}
\end{align}
In case of the internal cavity this is $\approx \SI{60}{GHz}$
With c - speed of light, $n\approx3.61$ - refractive index of the semiconductor and L - cavity length.

This function can be shifted by changing the temperature of the diode which can be accomplished by changing the direct current.
The direct current also effects the internal cavity in another way by changing the carrier concentration in the active layer. This reduces the optical path length (L) and effects the period width. Both effects don't have the same speed which results in so called 'mode hops' (further descriped below).

The external cavity acts similar as the internal cavity but now L is the distance from one end of the cavity to the diffraction grating. Here: $\nu \approx \SI{10}{GHz}$. By tilting the grating using the two knobs L can be modulated accordingly.
\begin{figure}[H]
	\includegraphics[width=0.6\textwidth]{../pics/gains.jpg}
	\caption{Gain functions that sum up to the total gain of the diode laser.}
	\label{fig:gains}
\end{figure}

\section{Tuning and modehops}
By modulating the gain functions listed above the wavelength of the emitted light can be changed. Figure \ref{fig:modeHop} shows how the gain functions add up together and how a so called mode hop occurs. As mentioned before the laser will always operate in the mode with the highest net gain. In figure \ref{fig:modeHop}a this is mode 'e0'. By shifting the internal cavity to the left, mode 'e-1' suddenly has the highest gain and the laser will start to operate in this mode (figure \ref{fig:modeHop}b). Moving further left the laser hops to mode 'e-2' (figure \ref{fig:modeHop}c) and then jumps all the way up to 'e3' because its gain is slightly bigger than 'e-3' (figure \ref{fig:modeHop}d).

\begin{figure}[H]
\includegraphics[width=0.6\textwidth]{../pics/modeHop.jpg}
\caption{Gain functions of grating feedback and external cavity got summed up and are fixed, while internal cavity changes. This results in 'mode hops'.}
\label{fig:modeHop}
\end{figure}

\section{Execution}
During the execution of the experiment the setup changes gradually and these intermediate steps are captured as a picture from the oscillograph. The power
of the laser is up to 30 mW and works at a wavelength around 780 nm. During the alignment safety goggles should be worn.
\subsection{Basic Settings and Laser Generating}
The Rubidium cell (RbC) gets heated up to 50 $^\circ$C by the controller (LC). Behind the focusing lense the light beam gets detected by an IR-card while
the current for the diode increases as can be seen in picture \ref{pic_setup1}. 
\begin{figure}[t]
 \includegraphics[width=\textwidth]{../pics/setup1.png}
 \caption{The laser beam, emitted by the diode gets detected by the IR card and can be seen with the camera.}
 \label{pic_setup1}
\end{figure}
With a camera the light spot can be seen on a TV monitor. Around the critical
voltage of 3.03-3.08 V a sharp slope in intensity of the light beam can be perceived which represents the transition of the beaming behaviour of the diode,
see also picture \ref{pic_stain}.
Below this critical voltage, the diode functions like an LED and above, stimulated emission dominates so that it serves as a laser. By slightly adjusting
the Top Knob and therefore the vertical orientation of the grating, the beam intensity gets maximized. the operating voltage is now a bit above the critical.
\begin{figure}
 \includegraphics[]{../pics/bigstain.jpg}
\end{figure}


\subsection{Placement of the Rumdidium cell}
Now we place the RbC in a way that the laser passes through. The camera gets placed so that it looks into the cell and a photo diode (PD) measures the 
amplitude of the beam as depicted in figure \ref{pic_setup2}.
\begin{figure}[t]
 \includegraphics[width=\textwidth]{../pics/setup2.png}
 \caption{The beam passes through the RbC. Fluorescence effects can be seen on the monitor. The PD detects the beam amplitude}
 \label{pic_setup2}
\end{figure}
The ramp generator (RC) implemented in the LC gets linked to the also implemented piezo modulator (PM). The output voltage of RC is depicted by an 
oszillograph as the upper curve in figure \ref{pic_peak1}. With a frequency of 10 Hz at the RC, the grating gets modulated by the PM and therefore the 
frequency of the laser beam.
Now appearances of fluorescence in the RbC can be seen as blinkings on the screen. This can be further achieved by modifying the Side Knob.

\subsection{The Spectrum of Rubidium}
We saw fluorescence in the RbC and now we can display the spectrum of Rubidium. To achieve this, a glass neutral density filter gets placed before the RbC and
a photo diode (PD) right behind it. The PD gets connected to the oscillograph which displays its voltage.
\begin{figure}[t]
 \includegraphics[width=\textwidth]{../pics/peaks.jpg}
 \caption{The output signal of the PM (upper) and of the PD (lower). Absorbtions and mode hoppings can be seen.}
 \label{pic_peak1}
\end{figure}
In figure \ref{pic_peak1} one can see that the beam is attenuated for certain frequencies. They can be identified in the following way. Light with a wavelength
$\lambda$ encounters the RbC. If it matches the exact difference between two energy levels the light gets absorbed by puting an electron from one level to
the other. By falling down to its former level a photon gets emitted isotropically and therefore the intensity of the original beam decreases.
With this method only the external resonater is modified by the PM and not the internal so that these jumps occur. So the laser does not pass through the 
frequencies continuously but skips at several ones.

\subsection{Simultaneous Modification of Current and PM}
To eliminate the jumps in the spectrum not only the PM but also the laser current will be modified. To do so, the modulation input gets connected with the 
RG. PM and current get modified simultaneously by the voltage produce by the RG. This means that now the external and internal resonator shift. Due to the 
increase (decrease) the beam intensity increases (decreases) the voltage of the PD shows a progression at which the absorbtion spectrum can be perceived
without jumps as in figure \ref{pic_peak1}.
\begin{figure}[H]
 \includegraphics[width=\textwidth]{../pics/CandPM.jpg}
 \caption{The laser current gets modified by the RG as well as the PM. The absorbtion lines can now be seen without hoppings.}
 \label{pic_CandPM}
\end{figure}

\subsection{Representation of the Bare Absorbtion Spectrum}
Finally we want to represent the absorbtion spectrum of Rubidium without the ramp offset. By use of a semipermeable mirror (50:50) which splits the laser beam
into two equivalently intense beams before it crosses the RbC as in figure \ref{pic_setup3}. This beam is detected by a second photo diode PDII which only
plots the ramp offset without the Rubidium spectrum, though. Now the LC serves a function to balance two incoming signals. The input of PDI and PDII gets
balanced by a regulator so that the common offset cancels out and only the bare absorbtion spectrum remains. This spectrum can be seen in figure 
\ref{pic_spectrum}
\begin{figure}[H]
 \includegraphics[width=\textwidth]{../pics/setup3.png}
 \caption{The 50/50 B.S. splits the beam. The signals of PDI and PDII differ by the ramp offset.}
 \label{pic_setup3}
\end{figure}

\begin{figure}[H]
 \includegraphics[width=\textwidth]{../pics/spectrum.jpg}
 \caption{The balanced signal is shown. The absorbtion spectrum of Rubidium can be seen clearly.}
 \label{pic_spectrum}
\end{figure}

\section{Discussion}
It can be said that with the very detailed manual it is quite easy to adjust a diode laser in a way that the absorbtion spectrum of such a sample as 
Rubidium can be probed and measured. A comparison of figure \ref{pic_spectrum} with a reference figure \ref{pic_ref} proves this ability and confirms that
the sample is indeed Rubidium.
\begin{figure}[H]
 \includegraphics[width=0.7\textwidth]{../pics/reference.jpg}
 \caption{Absorbtion spectrum of Rubidium \cite{ref}}
 \label{pic_ref}
\end{figure}


\begin{thebibliography}{WissOnl}
	\bibitem{Anl} TU Dortmund instruction for experiment Nr.60 \url{http://129.217.224.2/HOMEPAGE/Anleitung_FPBSc.html}
	\bibitem{ref} Laser spectroscopy of rubidium \\ \url{http://www.photodigm.com/literature/applications-notes/rubidium-absorption-spectroscopy}
\end{thebibliography}

% ========================================
%	Literaturverzeichnis
% ========================================

%\bibliographystyle{plainnat}			% Bibliographie-Style auswählen
%\bibliography{BIBDATEI}			% Literaturverzeichnis

% ========================================
%	Das Dokument endent
% ========================================
\end{document}
