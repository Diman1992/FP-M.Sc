% ========================================
%	Header einbinden
% ========================================

\input{longheader.tex}

% ========================================
%	Angaben für das Titelblatt
% ========================================

\title{V60 - Der Diodenlaser\\				% Titel des Versuchs 
	\large TU Dortmund, Fakultät Physik\\ 
	\normalsize Fortgeschrittenen-Praktikum}

\author{Jan Adam\\			% Name Praktikumspartner A
	{\small \href{jan.adam@tu-dortmund.de}{jan.adam@tu-dortmund.de}}	% Erzeugt interaktiven einen Link
	\and						% um einen weiteren Author hinzuzfügen
	Dimitrios Skodras\\					% Name Praktikumspartner B
	{\small \href{dimitrios.skodras@tu-dortmund.de}{dimitrios.skodras@tu-dortmund.de}}		% Erzeugt interaktiven einen Link
}
\date{06.06.2016}				% Das Datum der Versuchsdurchführung

% ========================================
%	Das Dokument beginnt
% ========================================

\begin{document}
	
% ========================================
%	Titelblatt erzeugen
% ========================================

\maketitle					% Jetzt wird die Titelseite erzeugt
\thispagestyle{empty} 				% Weder Kopfzeile noch Fußzeile

% ========================================
%	Der Vorspann
% ========================================

%\newpage					% Wenn Verzeichnisse auf einer neuen Seite beginnen sollen
%\pagestyle{empty}				% Weder Kopf- noch Fußzeile für Verzeichnisse

\tableofcontents

%\newpage					% eine neue Seite
%\thispagestyle{empty}				% Weder Kopf- noch Fußzeile für Verzeichnisse
%\listoffigures					% Abbildungsverzeichnis

%\newpage					% eine neue Seite
%\thispagestyle{empty}				% Weder Kopf- noch Fußzeile für Verzeichnisse
%\listoftables					% Tabellenverzeichnis
\newpage					% eine neue Seite


% ========================================
%	Kapitel
% ========================================

\section{Theoretical Essentials}
\section{Execution}
During the execution of the experiment the setup changes gradually and these intermediate steps are captured as a picture from the oscillograph. The power
of the laser is up to 30 mW and works at a wavelength around 780 nm. During the alignment safety goggles should be worn.
\subsection{}
\section{Discussion}
\begin{thebibliography}{WissOnl}
	\bibitem{Anl} TU Dortmund Versuchsanleitung zu Versuch Nr.57 (abgerufen am 20.4.2014) \url{http://129.217.224.2/HOMEPAGE/PHYSIKER/BACHELOR/FP/SKRIPT/V57.pdf}
\end{thebibliography}

% ========================================
%	Literaturverzeichnis
% ========================================

%\bibliographystyle{plainnat}			% Bibliographie-Style auswählen
%\bibliography{BIBDATEI}			% Literaturverzeichnis

% ========================================
%	Das Dokument endent
% ========================================
\end{document}
